\section{Planning vs Reinforcement Learning}\label{ref:solutions}

While often the possible states, actions and observations of a particular environment are easy to
define, it does not always happen that the transition, observation and reward functions are known in
advance.

When they are known, the solving process is called \textit{planning}, and consists in efficiently
looking through all possible transitions the agent could experience, and finding the best path
through them in order to maximize the return.

On the other hand, when they are not known, \textit{reinforcement learning} is used. In this case the
focus is on gathering knowledge about the environment while at the same time exploiting that
knowledge to obtain the maximum return. Reinforcement learning methods can range from approximating
the unknown functions from experience and then planning, to approximating the optimal value function
of the model and using it to obtain a policy, or even directly approximating the optimal policy.

In this work we are mostly interested in the problem of selecting subset of sensors, and so we
assume we have a complete model that we can use to plan.
\ys{you learn the model , not assume it}

\section{Offline vs Online}

An optimal policy can be applied at virtually no computational cost with the knowledge that the
behavior is constantly optimal. Computing such a solution in advance and using it later is called
\textit{offline planning}. The main disadvantage of offline planning is its very heavy upfront cost:
it needs to store all possible policies to be applied to any reachable belief at any horizon. This
is expensive not only in terms of space, but even more on computation, since any non-trivial POMDP
has an exponential explosion in the number of possible histories that an agent can experience and
each one of them has to be considered in order to evaluate the final policy. In addition, any small
change in the original model requires a full recomputation of the optimal policy, which limits the
amount of tuning that can be feasibly done on the model.

Offline planning methods generally leverage the Bellman equation and the recursive nature of the
value function; they plan in reverse temporal order, from the last steps the agent will take to the
first ones. This is because a policy for a certain horizon $h$ is always composed by a set of
actions to apply in certain beliefs which then leads to a policy for horizon of $h-1$. Thus, the
final policy needed is computed incrementally. All possible state, actions and observations are
considered, so as to provide the correct action in every possible situation.

In order to avoid the computational cost of offline methods, a new venue of approach was invented
called \textit{online planning}. In online planning the agent continuously plans for its current
belief, and nothing else. This requires constant computations, but at the same time makes possible
to tackle problems orders of magnitude bigger than what is possible with offline planning, and allow
for flexibility in updating the model, since there is no need to recompute a complete solution. In
addition, many online planning methods are \textit{any time}: this means that they can run for
whatever time is available, and they are always able to provide some sort of solution. The more time
is available, the more refined the solution will be. This is in contrast with offline methods, that
need to run to completion or otherwise nothing is gained.

In online planning computations go forward in time. From the current state/belief of the agent
possible futures are expanded in a tree of possibilities and rewards. In the end, the best action
for the current state is picked, the environment is advanced one step, and the process repeats
again.

In this work we chose to use an online approach, given the flexibility and power provided over
offline techniques, and because one of our goals was to scale to real-world scenarios, which are
generally intractable with offline approaches. An additional advantage which we will explore later
is that online planners are able to deal with non-PWLC value functions, which is a challenge when
using negative entropy as a reward function.

\section{Monte Carlo Algorithms}

At the time of writing, one of the most prominent ways of planning online is by using \textit{Monte
Carlo algorithms}. Monte Carlo algorithms are a class of randomized algorithms which use random
processes in order to compute their output. The output will be correct within a certain probability,
which trades off with their speed and their deterministic running time.

In the MDP/POMDP frameworks Monte Carlo algorithms only need results from samples of the
environment; as in, transition and reward results sampled from the model's distribution using
particular state-action pairs as inputs. Such a model is also called a \textit{generative model}. A
generative model $G$ fed with state $s$ and action $a$ would return a new state $s'$ and reward $r$
respecting the model probabilities:

\[ s', r \sim G(s, a) \]

An advantage of using such a model is that it many cases it is possible to obtain samples without
having the whole transition function for the model in an explicit form. The samples obtained can
then be used to approximate average returns, which in turns can be used to obtain an optimal policy.

\section{Monte Carlo Tree Search}

The most widely known algorithm using Monte Carlo techniques is called \textit{Monte Carlo Tree
Search} (MCTS). This algorithm is an MDP online planner which has received high praise for its
contributions in the development of highly performant board game AIs.

MCTS was originally devised as a method to approach the board game Go. The algorithm runs Monte
Carlo simulations from the current state of the agent, which is the input, and progressively builds
a tree of outcomes.

\begin{algorithm}
\begin{multicols}{2}
    \caption{Monte Carlo Tree Search}
    \SetKwFunction{mcts}{MCTS}
    \SetKwFunction{simulate}{Simulate}
    \SetKwFunction{rollout}{Rollout}

    \SetKwProg{main}{Algorithm}{}{}
    \main{\mcts{s}}{
        \KwData{State s}
\nl     \While{enough time is available}{
\nl         \simulate{\O, s, 0}\;
        }
\nl     \KwRet most valued action in tree root\;
    }

    \setcounter{AlgoLine}{0}
    \SetKwProg{myproc}{Procedure}{}{}
    \myproc{\rollout{s, d}}{
        \KwData{State s, Depth d}
\nl     r = 0\;
\nl     discount = 1\;
\nl     \While{$d <$ horizon}{
\nl         ($s', r'$) $\sim$ G($s$, random action)\;
\nl         r = r + discount * $r'$\;
\nl         discount = discount * $\gamma$\;
\nl         $s$ = $s'$\;
        }
\nl     \KwRet r\;
    }

    \setcounter{AlgoLine}{0}
    \SetKwProg{simul}{Procedure}{}{}
    \simul{\simulate{h, s, d}}{
        \KwData{History h, State s, Depth d}
\nl     select action $a$\;
\nl     ($s', r$) $\sim$ G($s,a$)\;
\nl     \If{$d < $ horizon}{
\nl         \eIf{T($h,a,s'$) is not \O}{
\nl             futureRew = \simulate{$(h,a,s'), s', d+1$}\;
            }{
\nl             initialize T($h,a,s'$)\;
\nl             N($h,a$) = V($h,a$) = 0\;
\nl             futureRew = \rollout{s', d+1}\;
            }
\nl         r = r + $\gamma$ * futureRew\;
            }
\nl     N($h,a$) = N($h,a$) + 1\;
\nl     V($h,a$) += $\frac{r-V(h,a)}{N(h,a)}$\;
\nl     \KwRet r\;
    }

\end{multicols}
\end{algorithm}

The tree starts out as root-only, where the root represents the current state of the agent. From
there, the algorithm selects an action. Initially the actions are selected randomly, but as more
simulations are performed MCTS starts to favor the actions with better value estimates, in order to
focus available computation time on promising branches. Once an action has been selected, MCTS uses
a generative model of the MDP to obtain a sample new state and reward.

If the new state already has a branch from the current node, MCTS follows the branch and repeats the
sample process again. If the new state does not have a branch, we attach a new single node to the
tree, and proceed in that direction. In MCTS a single node is added to the tree per simulated
episode; this is done in order to keep memory usage low, so that only the most often occurring paths will
be stored. Note that it is possible for MCTS to have multiple nodes representing the same state, but
in different histories. We do not consider the case where such nodes are merged together.

Once the sampling process arrives at a point where a new node needs to be added, the action
selection process changes to a default predetermined policy, which usually selects actions randomly
at uniform. This is done because MCTS has no information outside of the tree in order to do any
meaningful decision, and so it needs to rely on a predefined policy. Random actions are often useful
due to their lack of bias, and so they can give a better estimate of which paths are actually better.
When using the default policy, no new nodes are added to the tree, but the obtained reward outside
the tree is still recorded.

This process is repeated until a certain target horizon, or when a terminal state is reached. This
concludes a single sample episode. Once the episode is concluded, MCTS backpropagates obtained
rewards up the chain, and each node value is computed as the average of all obtained returns from
paths down from it. This allows for a new estimation of action values at any point in the tree
branch that has been explored.

At this point, a new sample episode is started from the root of the tree. This process is repeated
until there is time left.

A nice property of MCTS is that it is guaranteed to converge to the optimal policy as long as the
most promising actions are given a higher probability of being chosen than the others. This is
because in the long run the value averages will converge to the values of the actions taken most
often.  In addition, since it builds a tree, part of the work done can be reused on the next time
step. Once a move is tried in the real world, the tree root can be moved to the new state perceived,
while the rest of the tree is discarded.


\section{Upper Confidence Bounds for Trees}

A technique which has recently gained popularity to exactly determine which actions should be chosen
within the tree and which ratio should be used between promising actions and the rest is called
\textit{Upper Confidence Bounds for Trees} (UCT). UCT tries to empirically derive a variance
estimate from the data, sorting actions by their upper bounds on value.

The upper bound estimate is composed of two parts: the currently estimated value of the action $v$,
and the estimated UCT variance. Thus the value that UCT assigns at a given action $a$ is:

\[ UCT(a) = v + c*\sqrt{\frac{\log{N}}{n}} \]

Where $N$ is the number of times the node in the tree currently being considered has been visited,
and $n$ is the number of times action $a$ has already been tried from the node. $c$ is a constant
which is dependent on the problem, which defines the amplitude of the variance, which needs to be in
tune with the magnitude of the rewards of the MDP. This number is generally tuned manually, and is
around the maximum return that can be obtained by the agent.

In this way actions that have not been tried for a long time still have a chance of being selected,
which prevents hard pruning in the tree which may prevent a better solution to be discovered.

\section[]{Partially Observable Monte-Carlo Planning%
\sectionmark{POMCP}}
\sectionmark{POMCP}

\cite{cit:pomcp} extended the MCTS framework to POMDPs, creating one of the best POMDP online
solvers available today, \textit{Partially Observable Monte-Carlo Planning}(POMCP). This algorithm
incorporates the advantages of MCTS while adding support for belief states and observations.

The trouble of naively applying MCTS to a belief MDP extracted from a POMDP is its necessity to
sample directly from state to state. While this can be a cheap operation in MDPs, requiring
samples from relatively simple distributions, computing a new belief given a previous belief and
observation in a POMDP, called \textit{belief update}, can be a very expensive operation. Recall that a belief
update can be performed using the following formula:

\[ b'(s') = \frac{O(s', a, o)\sum_{s\in S}T(s,a,s')b(s)}{Pr(o|a,b)} \]

For a full belief, this requires $O(|S|^2)$ operations, which for POMDPs with a large state space
can result in an extremely expensive operation. Since MCTS requires a very high number of samples in
order to provide useful answers, each sample needs to be performed in as little time as possible.

POMCP solves this problem by representing beliefs as particle vectors containing a certain amount of
state particles. Such a \textit{particle belief} can be created by repeatedly sampling a belief, and
gathering all samples in a vector - not a set.

\begin{algorithm}
    \caption{Partially Observable Monte-Carlo Planning}
\begin{multicols}{2}
    \SetKwFunction{pomcp}{POMCP}
    \SetKwFunction{simulate}{Simulate}
    \SetKwFunction{rollout}{Rollout}

    \SetKwProg{main}{Algorithm}{}{}
    \main{\pomcp{b}}{
        \KwData{Belief b}
\nl     \While{enough time is available}{
\nl         s $\sim$ b\;
\nl         \simulate{\O, s, 0}\;
        }
\nl     \KwRet most valued action in tree root\;
    }

    \setcounter{AlgoLine}{0}
    \SetKwProg{myproc}{Procedure}{}{}
    \myproc{\rollout{s, d}}{
        \KwData{State s, Depth d}
\nl     r = 0\;
\nl     discount = 1\;
\nl     \While{$d <$ horizon}{
\nl         (s', r') $\sim$ G(s, random action)\;
\nl         r = r + discount * r'\;
\nl         discount = discount * $\gamma$\;
\nl         s = s'\;
        }
\nl     \KwRet r\;
    }

    \setcounter{AlgoLine}{0}
    \SetKwProg{simul}{Procedure}{}{}
    \simul{\simulate{h, s, d}}{
        \KwData{History h, State s, Depth d}
\nl     select action $a$ with UCT\;
\nl     ($s', o, r$) $\sim$ G($s,a$)\;
\nl     \If{$d < $ horizon}{
\nl         \eIf{T($h,a,o$) is not \O}{
\nl             futureRew = \simulate{$(h,a,o), s', d+1$}\;
            }{
\nl             initialize T($h,a,o$)\;
\nl             N($h,a$) = V($h,a$) = 0\;
\nl             futureRew = \rollout{s', d+1}\;
            }
\nl         r = r + $\gamma$ * futureRew\;
            }
\nl     B(h) = B(h) $\bigcup \;\{ s' \}$\;
\nl     N($h,a$) = N($h,a$) + 1\;
\nl     V($h,a$) += $\frac{r-V(h,a)}{N(h,a)}$\;
\nl     \KwRet r\;
    }

\end{multicols}
\end{algorithm}

POMCP starts its execution with a belief state as input, just as MCTS started with a single state as
input. It converts this belief into a particle belief. Then, for each sample episode, it samples a
state directly from the particle belief. This state is then used, together with an action, to
generate a new possible state, observation and reward. The observation is the one determining the
next node that the sample episode continues to.

This next node is itself associated with a particle belief: every time this node is visited, its
respective particle belief incorporates the new sample state that was generated together with the
observation. This constitute a cheap belief update which is performed by the generative model over
the previous particle belief to the current particle belief.

One of the drawbacks of particle beliefs is the fact that, being approximate representations of the
true underlying beliefs, over time they can be affected by particle deprivation. This can be
resolved via particle reinvigoration, and its actual effect on the solution is dependent on the type
of problem being solved.

Once in the new node, the state that has just been added to the node's particle belief is then
combined with a new action to generate the next step state, observation and reward. Rewards are used
in the same way as in the MCTS algorithm, providing a value for all nodes and actions that are in
the tree. Actions are again selected using UCT. The episode continues up until a leaf, a single node
is added, and then a default policy is used to complete the episode.

When the time allotted to POMCP is over, the algorithm returns its best guess for the optimal
action. Once the action is used in the real world, POMCP can obtain an observation and prune its
internal tree, without having ever the need to perform a true belief update.
