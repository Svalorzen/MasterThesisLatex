\section{$\rho$-POMCP}

While the POMCP algorithm is very powerful, unfortunately it cannot be directly applied to
$\rho$POMDPs. This is because its exploration strategy through the action-observation tree relies
on reward functions which depend on states only. Since in a $\rho$POMDP the reward function depends
on beliefs rather than states, it is necessary to modify the algorithm in order to be used on Active
Perception tasks.

In this Section we present a modification of the algorithm POMCP which is able to deal with a
belief-based reward function. We discuss in more details why POMCP cannot be applied directly to
$\rho$POMDPs in Section \ref{ref:rewestimation}.

We call our POMCP variant $\rho$-POMCP. An important fact to note is that $\rho$-POMCP is able to deal
with any $\rho$POMDP, and not only with Active Perception tasks. For example, this algorithm can be
applied to problems where the agent can influence the state of the world, and also (with minor
modifications to reintroduce state dependent rewards) when the reward function is both dependent on
states and belief.

% \subsection{Non-PWLC Value Function}
%
% Negative entropy is not a PWLC function. This means that the value function of a $\rho$POMDP which
% uses it as reward function will not be PWLC too \cite{cit:rpomdp}. This is a problem for exact
% offline planning since usual solving techniques rely on the PWLC property in order to detect beliefs
% where the currently computed optimal policy is not actually optimal. Once such a belief point is
% discovered (called \textit{witness point}), the optimal policy is updated to be optimal in that
% belief too.
%
% This problem is approached generally via approximation. With $\rho$-POMCP, we directly approximate
% rewards for the possible futures of the current situation, without the need to solve a POMDP
% completely for any possible beliefs. Thus, witness points do not need to be discovered, and we can
% ignore the PWLC property.
%
% \ys{I dont get the point of this subsection}

\subsection{Reward Estimation}\label{ref:rewestimation}

The main challenge when applying POMCP to a $\rho$POMDP is that the algorithm can only extract
reward information from a generative model of the environment. Since a belief is an abstraction
created by the agent, a generative model containing a belief-dependent reward function cannot exist.

A possible approximate solution involves converting the belief-dependent reward function into a
state-based reward function, approximating the $\rho$POMDP problem back to a POMDP \cite{cit:rpomdp}.
The main drawback of this approach is that it requires a very high number of additional actions in
the approximated model in order to adequately describe the original model. This in turn
significantly slows down online planning approaches, since their performance is directly bound to
the size of the tree to explore.

Instead, in our approach we estimate the immediate rewards within the tree from maximum-likelihood
belief estimates computed from POMCP's particle beliefs. However, since all particle beliefs are
updated constantly, continuously updating all ML estimates would excessively hamper performance. We
solved this problem differently for estimating max of belief rewards and negative entropy rewards.

For estimation of max of belief, $\rho$-POMCP simply keeps track of the number of particles in the
particle belief $N(b)$, the number of particles for each state $N(s)$ and the most common
state $max_s$. At each particle $k$ insertion $N(k)$ is compared with $N(max_s)$. Whichever is higher
determines the new most common particle, either $k$ or the old $max_s$. Finally, the new max of
belief is computed using maximum likelihood as $N(max_s)/N(b)$.

\begin{algorithm}[H]
    \caption{Max of Belief Reward Estimation}
    \SetKwFunction{update}{Update Estimate of $\rho(b)$}

    \SetKwProg{main}{Algorithm}{}{}
    \main{\update}{
        \KwIn{Particle Belief b, Particle s}
\nl     N(s) = N(s) + 1\;
\nl     \If{N(s) $>$ N(max\_s)}{
\nl         max\_s = s\;
        }
\nl     $\rho(b)$ = $\frac{N(max\_s)}{N(b)}$\;
    }
\end{algorithm}

For estimation of entropy, the situation is not quite so straightforward. Entropy estimation from
discrete samples is bound to be biased, and there is no way to remove this bias completely
\cite{cit:badentropy}. On the other hand, techniques to decrease bias are computationally heavy and
rely on usage of additional complex distributions to account for bias \cite{cit:entropyfixes}.

In practice bias becomes small as long as the number of samples is high. Thus, we kept our
estimation algorithm as simple as possible in order to maximize the number of samples that can be
done in any given time. We compute negative entropy as a summation of terms, one for each type of
particle. We thus approximate the entropy, whose formula is:

\[ -H(b) = \sum_s p(s) \log p(s) \]

With the following computation:

\[ -\hat{H}(b) \approx \sum_s \frac{N(s)}{N} \log \frac{N(s)}{N} \]

Where $N$ is the total count for all states encountered. $\rho$-POMCP stores internally both the
latest negative entropy estimate $-\hat{H}(b)$, and each of the separate terms that compose it. At each
insertion in a particle belief of a new particle $k$, we simply remove from our previous full estimate the
relevant term for $k$, recompute the term and add it again. This approach results in an approximate
estimator which in practice is good enough as long as particles of every state are added with relative frequency.

\begin{algorithm}[H]
    \caption{Negative Entropy Reward Estimation}
    \SetKwFunction{update}{Update Estimate of $\rho(b)$}

    \SetKwProg{main}{Algorithm}{}{}
    \main{\update}{
        \KwIn{Particle Belief b, Particle Type s}
\nl     $\rho(b)$ = $\rho(b)$ - $\rho(s)$\;
\nl     N(s) = N(s) + 1\;
\nl     p = $\frac{N(s)}{N(b)}$\;
\nl     $\rho(s)$ = p $\cdot$ log(p)\;
\nl     $\rho(b)$ = $\rho(b)$ + $\rho(s)$\;
    }

\end{algorithm}
% \ys{indicate input and output of the algorithm ... also is there a step like 'for each particle
% $s$' missing from the algorithm?? -  same comment for the algorithm above}
%
% The input is the KwData.. the output I think it's self-explanatory, since the algorithm is used to
% estimate the reward, the output is \rho(b)..
% Also no there's no for each particle, that's kind of the point to keep the whole thing fast.
% Otherwise it'd take a lot of time to update estimates. The idea is use only the newest particle in
% order to update the estimate.

% \ys{can you cite something to support this?}
%
% Eugenio: Not directly, aside from the law of large numbers.. since in the limit each state will be
% visited an infinite number of times, the approximation for each \rho(s) will converge to its true
% value, thus making \rho(b) converge too.. but I don't have anything specific to write this yet. If
% you want I can just remove the claim I guess..

\subsection{Value Backup}

A second problem is dependent on how returns are calculated in POMCP. $Q^\pi_h(b,a)$ is computed as the
sum of two different factors: the immediate reward $\rho(b,a)$ plus a discounted term representing
the expected return from then on.

\[ Q^\pi_h(b,a) = \rho(b,a) + \gamma \cdot \sum_{o\in \Omega} O(o|b,a) V^\pi_{h-1}(b') \]

POMCP tries to approximate the value of $\rho(b,a)$ as computed in Equation \ref{rhoeq} by averaging
all results from $R(s,a)$ done using particles extracted from the all beliefs and actions
encountered during sample episodes.

\[ \hat{Q}^\pi_h(b,a) \approx \frac{\sum_{s \in S} n_s R(s,a)}{N} + \gamma \cdot \frac{\sum_{o\in \Omega} n_o f_o}{N} \]

Where $N$ is the number of times that action $a$ has been taken in belief $b$, $n_s$ the number of
times state $s$ has been sampled from $b$ when taking action $a$, $n_o$ is the number of times that
observation $o$ has been seen from $b$ and $a$ and $f_o$ is the average of all returns sampled after
observation $o$. As with MCTS, as long as the better actions are chosen more often than the others,
this procedure will converge to the true expected return.

However, in our case, the reward function is different from \ref{rhoeq}. This is because we do not have $R$
anymore. Instead, we need to base our definition for an immediate reward on the only function we have, 
which is $\rho(b)$. The definition we use is the following:

\[ \rho(b,a) = \sum_{o\in \Omega} O(o | b, a) \rho(b') \]

We assign an immediate reward to an action based on the value of all beliefs reachable multiplied by the 
probability of reaching them. The value of the possible next beliefs $\rho(b')$ is then extracted from our estimation. Thus, 
the value for a belief-action pair becomes:

\[ Q^\pi_h(b,a) = \sum_{o\in \Omega} O(o | b,a) \rho(b') + \gamma\sum_{o\in\Omega} O(o|b,a) V^\pi_{h-1}(b') \]

The problem is that our method of estimation for $\rho(b')$ does not allow a direct back-propagation
of each $\rho$ estimation back up the chain through $V$, as it happened in POMCP via samples. This
is because we are not approximating rewards as the average of simulations, but by improving our
estimates of the beliefs. So after a new reward estimate for a given beliefs, all previous estimates
are effectively useless and need to be discarded.

This means that whenever a new estimate for any given $\rho(b)$ is computed, $\rho$-POMCP needs to
substitute the previous estimate all the way up the chain, and cannot simply average in the new
estimate.

We solved this problem by keeping the average mechanism present in POMCP, with the difference that
we average together fake datapoints built specifically to replace previous estimates. For
example, suppose we have a particular estimate $r$ for $\rho(b, a)$.

% \ys{it can be bit confusion going back and forth between $\rho(b)$ and $\rho(b,a)$. Once you have
% defined $rho$ to be dependent only on $b$, stick to it or else the text does not come out as
% consistent}
%
% Actually, the two definitions are different. I have defined both \rho(b), and \rho(b,a), and I'm
% using them in different ways, just as Q and V are different.

\[ \rho(b,a) \approx r \]

We can decompose this into all our estimates for all beliefs reachable from $b$ and $a$:

\[ \rho(b,a) \approx \frac{\sum_{o\in\Omega} n_o r_o}{N} \]

Suppose now we experience $b$ and $a$ again, and obtain observation $\tilde{o}$. We update our new
reward estimate for the new belief into $r'_{\tilde{o}}$. If we averaged normally, we would get:

\[ \rho(b,a) \approx \frac{ n_{\tilde{o}} r_{\tilde{o}} + r'_{\tilde{o}} +
\sum_{o \neq \tilde{o} \in \Omega} n_o r_o}{N+1} \]

This is not what we want, however, since as you can see the previous estimate for observation
$\tilde{o}$, $r_{\tilde{o}}$, is still present in the equation even though we now consider it
obsolete.

Instead, suppose we average in $n_{\tilde{o}}(r'_{\tilde{o}} - r_{\tilde{o}}) + r'_{\tilde{o}}$:

\[ \rho(b,a) \approx \frac{ n_{\tilde{o}} r_{\tilde{o}} + n_{\tilde{o}}(r'_{\tilde{o}} -
r_{\tilde{o}}) + r'_{\tilde{o}} +
\sum_{o \neq \tilde{o} \in \Omega} n_o r_o}{N+1} \]

\[ \rho(b,a) \approx \frac{ n_{\tilde{o}} r_{\tilde{o}} + n_{\tilde{o}}r'_{\tilde{o}} -
        n_{\tilde{o}} r_{\tilde{o}} + r'_{\tilde{o}} +
\sum_{o \neq \tilde{o} \in \Omega} n_o r_o}{N+1} \]

\[ \rho(b,a) \approx \frac{ ( n_{\tilde{o}}+1) r'_{\tilde{o}} +
\sum_{o \neq \tilde{o} \in \Omega} n_o r_o}{N+1} \]

Which is what we wanted, since now the previous estimate for $\tilde{o}$ has disappeared, completely
replaced by the updated one, $r'_{\tilde{o}}$. Similarly, we can create fake reward points to backup
in the tree so that any $V(b)$ can be substituted with the more up-to-date $V'(b)$. The value backed
up would be:

\[ N ( V'(b) - V(b) ) + V'(b) \]

% \ys{I am not sure what is going on here, you might have to explain it better}
%
% Hope this is more clear. We simply have to remove the old estimate from the values stored in the
% tree, since once we have a newer one the older one is wrong. This is simply a way to do it.

\subsection{Max vs Mean}

The POMCP algorithm averages together all sampled rewards, and depends on UCT to let the tree
converge to the true optimum. This works since UCT samples the best actions infinitely more often in
the limit, which lets the final estimated value for it converge to the true value. However, this
procedure needs to sample many more times to make up for the fact that, at first, results for
suboptimal actions are averaged into estimates too.

POMCP inherits this behavior from the original MCTS algorithm. However, in the original paper from
2006 a number of different approaches to merge together backed up values were tried
\cite{cit:mcts}. The authors decided to stick with using the mean operator to backup values since,
with the computational power available at the time, it gave better and more consistent results.
However, the authors also demonstrated that if enough samples are available, using a max operator
over actions actually improves performance.

We implemented both backup methods for $\rho$-POMCP in order to determine which one would perform
best under different conditions.

\subsection{Pseudocode}

Here we describe the full $\rho$-POMCP algorithm in more detail.

\begin{algorithm}[H]
    \caption{$\rho$-Partially Observable Monte-Carlo Planning}
\begin{multicols}{2}
    \SetKwFunction{pomcp}{$\rho$-POMCP}
    \SetKwFunction{simulate}{Simulate}

    \SetKwProg{main}{Algorithm}{}{}
    \main{\pomcp}{
        \KwIn{Belief b}
\nl     \While{enough time is available}{
\nl         s $\sim$ b\;
\nl         \simulate{\O, s, 0}\;
        }
\nl     \KwRet $\arg\max_a V(a)$\;
    }

    \setcounter{AlgoLine}{0}
    \SetKwProg{simul}{Procedure}{}{}
    \simul{\simulate}{
        \KwIn{History h, State s, Depth d}
\nl     N($h$) = N($h$) + 1\;
\nl     select action $a$ with UCT\;
\nl     ($s', o$) $\sim$ G($s,a$)\;
\nl     \eIf{T($h,a,o$) is \O}{
\nl         initialize T($h,a,o$)\;
\nl         newNode = true\;
        }{
\nl         newNode = false\;
        }
\nl     B($h,a,o$) = B($h,a,o$) $\bigcup \;\{ s' \}$\;
\nl     \eIf{$d < $ horizon and newNode == false}{
\nl         r = \simulate{$(h,a,o), s', d+1$}\;
        }{
\nl         N($h,a,o$) = N($h,a,o$) + 1\;
\nl         r = $\rho(h,a,o)$\;
        }
\nl     N($h,a$) = N($h,a$) + 1\;
\nl     V($h,a$) = V($h,a$) + $\frac{r - V(h,a)}{N(h,a)}$\;
\nl     oldV = V($h$)\;
\nl     V($h$) = $\rho(h)$ + $\gamma \cdot \max_{a'} V(h,a')$\;
\nl     \KwRet N($h$) * ( V($h$) - oldV ) + V($h$);
    }
\end{multicols}
\end{algorithm}

$\rho$-POMCP starts off as POMCP, by simulating episodes from the root of the tree.

Starting from the root, $\rho$-POMCP samples a single state $s$ from the root's particle belief.
Then, using UCT, it selects a state $a$, and uses it with $s$ to sample a new state $s'$, a reward
$r$ and an observation $o$. The root child representing the sequence $(a,o)$ is selected, or if it
did not exist, it is created. The algorithm then updates the particle belief of the child before
deciding whether to continue to descend the tree. While descending, POMCP appends $s'$ to the
particle belief of the selected child node. Then, it repeats the action selection and sampling
process using $s'$ as its $s$. This repeats until the algorithm is unable to descend further into
the tree, at which point the future reward is approximated as simply the reward of the last
selected child's belief.

The algorithm then updates the value of all actions taken during the descent, while backing up fake
datapoints to update all old reward estimates up in the tree.

Notice that $\rho$-POMCP lacks a rollout procedure. This is because $\rho$-POMCP has no way to
approximate rewards when outside the tree, and so skips random policy rollouts altogether.  A visualization of the process can be seen in Figure \ref{ref:rpomcpimg}.

\begin{figure}[ht!]
\centering
\begin{tikzpicture}[->,>=stealth',shorten >=1pt,auto,node distance=4cm,thick,main node/.style={circle,draw,font=\Large\bfseries}]
\tikzstyle{state} = [circle, minimum size=.7cm, draw=black, fill=green!30]
\tikzstyle{empty} = [coordinate,draw=none, fill=none]
\tikzstyle{arrow} = [thick,->,>=stealth]

\node (particleb) at (0,.8) [draw=none] {[\hl{$s_1$},$s_2$,...,$s_k$]};
\node (b)  at (0,0)  [state] {$b$};
\node (a1) at (0,-2) [empty] {};
\node (a2) at (1.5,-2) [empty] {};
\node (a3) at (3,-2) [empty] {};
\node (a4) at (4.5,-2) [empty] {};
\node (b1) at (0,-5) [state] {$b'$};
\node (b2) at (1.5,-5) [state] { };
\node (b3) at (3,-5) [state] { };
\node (b4) at (4.5,-5) [state] { };
\node (particleb1) at (0,-5.8) [draw=none] {[\hl{$s'$},...]};
\node (g) at (-2.2,-2.3) [draw=none,rotate=90] {$G(s,a)$};
\node (re) at (-3, -5) [align=center,draw=none,rotate=90] {Reward\\Estimation};
\node (null) at (0, -7.6) [align=center,draw=none, minimum width=1.35cm, minimum height=.7cm] {...};

\path
     (b)  edge [] node [left,pos=0.7] (aa) {\hlc[Dandelion]{$a_1$}} (a1)
          edge [] node [left,pos=0.7] {$a_2$} (a2)
          edge [] node [left,pos=0.7] {$a_3$} (a3)
          edge [dashed] node {} (a4)
     (a1) edge [] node [left,pos=0.6] (oo) {\hlc[SkyBlue]{$o_1$}} (b1)
          edge [] node [left,pos=0.6] {$o_2$} (b2)
          edge [] node [left,pos=0.6] {$o_3$} (b3)
          edge [dashed] node {} (b4);
          
\draw [draw=none] (particleb1) -- node [midway] (tmp) {} (null);

\draw (particleb1) -- (null);
\draw (tmp) -| (re);
\draw (null) -| (re);

\draw (re) |- node [above,pos=0.2,rotate=90] {\hlc[LimeGreen]{ $r$ }} (b);
\draw (re) -- node [above,pos=0.6] {\hlc[LimeGreen]{ $r'$ }} (b1);

\draw [-,line width=5pt,draw=white] (g.south) |- (particleb1);

\draw [*->] (g.south) |- (oo);
\draw [-,line width=5pt,draw=white] (g.south) |- (oo);
\draw [*->] (g.south) |- (oo);

\draw (g.south) |- (particleb1);

\draw [-,line width=5pt,draw=white] (aa) -| (g.south);
\draw (aa) -| (g.south);
\draw (particleb) -| (g.south);


\end{tikzpicture}
\caption{A graphical representation of a single $\rho$-POMCP step. A particle state $s_1$ is chosen randomly from the root and an action $a_1$ is chosen with UCT. They generate an observation $o_1$ and a new state $s'$ which are used to update the next belief node in the tree (or to add a new one if needed). The new particle belief is then used to estimate the new reward $r$ for the transition, which replaces previous estimates. We repeat the process starting from $s'$ until we add a single node to the tree, and then we restart from the root with no rollout phase.}
\label{ref:rpomcpimg}
\end{figure}

\section{Multiple Target Tracking}

One of the main limiting factors for Monte-Carlo online methods is the branching factor of the
tackled problem. While MCTS, POMCP and $\rho$-POMCP can deal with incredibly large state and
observation spaces, they cannot deal easily with a problem that leads to a vast number of branches
to be explored. This is because each new branch significantly increases the number of sample
episodes that need to be performed in order to obtain an accurate estimate for the best action.

The main branching factor in the problem we are considering, which is multi-sensor systems, is the
number of targets to be tracked. This is because each target moves simultaneously, and thus from a
given state there are many more possible states reachable. Thus, the branching factor for a problem
is exponential in the number of targets.

In order to avoid dealing with this problem, we setup a system where we run concurrently many
$\rho$-POMCP instances. We assume that each target moves independently, and that targets do not
affect observations of each other. Each solver computes an individual policy for a single target
only and receives individual observations. This allows each solver to work on a relatively small
tree, without exploding the number of branches. In addition, this allows to work with a variable
number of targets, since it possible to dynamically adjust the number of online planners running at
any time, depending on the number of targets present in the environment.

We then estimate the final action to be performed by merging the separate results of all the
solvers. From each solver we obtain a value estimate for each available action at the root of the
tree; the final action is then the one that maximizes the sum of all values across solvers. In order
to obtain consistent evaluations of all root actions so that we can compare them between solvers, we
disable UCT for the first depth level in the tree, and instead sample equally from each action.

Once a final action is chosen it is applied, and all trees in all solvers are pruned consequently.
The resulting observation is fed on all solvers, and finally a new round of planning takes action,
in order to act in the next timestep.

% \section{Multiple Camera Selection}
%
% In the same way that increasing the number of targets influences the branching factor of the
% tracking problem, allowing the agent to concurrently select a subset of available cameras at each
% timestep makes the problem increasingly harder for $\rho$-POMCP. This is because the algorithm needs
% to consider all possible subsets of the available cameras in order to determine the best one.
%
% As for the other problems we have discussed, the most common solution is approximation. A possible
% method to approach this problem is submodularity. Submodularity evaluates actions singularly in a
% multi step selection process; at each step the best action is added to the subset, and then all
% remaining actions are re-evaluated taking into account that previously added actions will now be
% performed. This avoids selecting overlapping cameras that, while by themselves would provide much
% information, would not be optimal if used together due to redundant information. This solution is
% explored by \cite{cit:relworksatsangi}.
%
% In order to focus more directly on the performance of the $\rho$-POMCP algorithm, in this work we
% force the agent to select a single camera at each timestep, avoiding the multiple camera selection
% problem.
% \ys{this subsection can go in future work/discussion, you can rather expand on multi-person tracking, if you wish. Make sure to point that we \textit{assume} that people walk independently of each other and they do not affect observations of each other}
