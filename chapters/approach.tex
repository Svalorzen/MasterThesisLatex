\section{Myopicity of Active Perception}

As we have already said, POMDPs can be used naturally to compute non-myopic policies to solve a
particular task. However, in order to justify the usage of POMDPs, we have to demonstrate that the
Active Perception task requires the usage of non-myopic solutions in the first place.  This is
because not all problems necessarily require non-myopic solutions.

\ys{reword this statement, we dont have to justify usage of POMDPs, its should be more along the lines of, we believe pomdp provides a natural framework for modelling this kind of task ( which we have already conveyed to user in introduction and background) }

We propose two different POMDP models of Active Perception problems. For one of the models we will
prove that a non-myopic solution is required in order to obtain an optimal solution. For the second
we will prove that a non-myopic solution is not required. The results we show are for both max of
belief and negative entropy reward functions.

\ys{Again, rather write this para as we have tried myopic and non-myopic approaches in different settings. In some settings, myopic policy proved to be as optimal as nonmyopic, where as in some settings non myopic policy has a clear advantage over myopic. We discuss the nature of this phenomenon in detail later.}
\ys{why are models still in approach?}

While we were not able to demonstrate a general way to show whether any particular problem will need
non-myopicity in the solution, the fact that Active Perception tasks exist which require
non-myopicity is sufficient to motivate our choice of POMDPs in this work.

\subsection{Non-Myopic World}

Let's consider a world where the agent is tasked with tracking a single person that is walking. The
person can, at any moment, be in one of a series of rooms, and can transition between them. In each
room there is a camera which can observe the room itself and nothing else. If the agent activates
the camera situated in a room the person just entered there is a $0.8$ probability that the agent will
obtain a ``found'' observation, and a $0.2$ probability that it will obtain a ``null'' observation.
On the other hand, if the agent activates any other camera, it will obtain a ``null'' observation
with $0.8$ probability and a ``found'' observation with $0.2$ probability. Each time a camera is
activated, the person will move to another accessible room. The agent's goal is to guess correctly
where the person has moved for two times in a row, knowing only that at the start the person could
be in any room.

We will now create an instance of the problem we have just finished describing. Let's now suppose
that in our world there are 6 rooms, and we know that the person will transition from one to another
with the following probabilities:

\begin{center}
\begin{tikzpicture}[->,>=stealth',shorten >=1pt,auto,node distance=4cm,thick,main node/.style={circle,draw,font=\Large\bfseries}]
\tikzstyle{state} = [circle, draw=black, fill=green!30]
\tikzstyle{arrow} = [thick,->,>=stealth]

\node (s1) at (0,0) [state] {S1};
\node (s2) at (2,3) [state] {S2};
\node (s3) at (4,0) [state] {S3};
\node (s4) at (6,0) [state] {S4};
\node (s5) at (8,3) [state] {S5};
\node (s6) at (10,0) [state] {S6};

\path
    (s1) edge [loop below] node {0.333} (s1)
         edge [bend left] node {0.333} (s2)
         edge [bend left=10] node[below] {0.333} (s3)
    (s2) edge [loop above] node {0.333} (s2)
         edge [bend left=10] node[above,rotate=60] {0.333} (s1)
         edge [bend left] node {0.333} (s3)
    (s3) edge [loop below] node {0.333} (s3)
         edge [bend left=10] node[above,rotate=-60] {0.333} (s2)
         edge [bend left] node {0.333} (s1)
    (s4) edge [bend left] node {1.0} (s5)
    (s5) edge [bend left] node {1.0} (s6)
    (s6) edge [bend left] node {1.0} (s4);

\end{tikzpicture}
\end{center}

As we said, at first the agent does not know where the person is at all. The myopic expected return
for the first step only is:

\[ E[\text{return}| a_1 ] = \sum_{o\in \Omega} p(o | b, a_1) \cdot \rho(b') \]

Where $b'$ is the result of the belief update of $b$ with $o$ and $a_1$. The non-myopic expected return over
both steps is:

\[ E[\text{return}| a_1, a_2] = \sum_{o \in \Omega} p(o | b, a_1) \left ( \rho(b') + \sum_{o' \in
\Omega} p(o'| b', a_2) \cdot \rho(b'') \right ) \]

Where $b''$ updates from $b'$, $o'$ and $a_2$. It is easy to show that, both for max of belief and
negative entropy as $\rho$, all actions in the presented problem have the same myopic one-step
expected return from a uniform belief. This is done by simply computing the previously described
summatory over every term and performing the appropriate belief updates.

However, expected non-myopic returns are not the same for all choices of $a_1$ and $a_2$. For
example selecting $a_1 = S1$ always results in a certain expected return $v$, independently from
$a_2$. On the other hand, if $a_1 = S4$, all expected returns are greater than $v$ for all but one
choice of $a_2$. In other words, selecting $a_1 = S1$ leads necessarily to a lower two-step expected
return, since the agent can select $a_2$ to only pick the best cases.

Selecting $a_1$ to avoid the worst cases can only be done when there is a way to distinguish between
two different $a_1$.  Since myopic expected returns are the same for all $a_1$, this is impossible
to do with a myopic solution.

Note that the problem considered can be extended into uncountably many models where the same
properties apply. This could be done for example by adding a certain number of rooms that the person
needs to go through before ending up in the model presented, and so on. Thus a non-myopic solution
is required for uncountably many Active Perception tasks.

\ys{I feel this whole section about models and consequences should be in experiements and the consequence or explanation about what is happening should be in discussion or experiments.}

\subsection{Myopic World}

A different instance of the previous problem is one where there are only 4 rooms, and the person
will transition from state to state with the following probabilities:

\begin{center}
\begin{tikzpicture}[->,>=stealth',shorten >=1pt,auto,node distance=4cm,thick,main node/.style={circle,draw,font=\Large\bfseries}]
\tikzstyle{state} = [circle, draw=black, fill=green!30]
\tikzstyle{arrow} = [thick,->,>=stealth]

\node (s1) at (0,0) [state] {S1};
\node (s2) at (0,4) [state] {S2};
\node (s3) at (4,4) [state] {S3};
\node (s4) at (4,0) [state] {S4};

\path
    (s1) edge [bend right] node {0.5} (s4)
         edge [bend left] node {0.5} (s2)
    (s2) edge [bend left] node {0.5} (s1)
         edge [bend right] node {0.5} (s3)
    (s3) edge [bend right] node {0.5} (s2)
         edge [bend left] node {0.5} (s4)
    (s4) edge [bend left] node {0.5} (s3)
         edge [bend right] node {0.5} (s1);

\end{tikzpicture}
\end{center}

Once again, we can compute both myopic and non-myopic expected returns. This time, no matter the
choice of actions, there is no particular incentive for an agent to select a pair over another. If
$\rho$ is negative entropy, certain pair of actions have a higher expected return than others. But
for each $a_1$ there is a $a_2$ which obtains the maximum expected return possible.

Thus, non-myopic in this particular model has no particular advantage over a more simple myopic
solution. Once again this problem can be extended into uncountably infinite examples of Active
Perception tasks.

\section{$\rho$-POMCP}

In this Section we present a modification of the algorithm POMCP which is able to deal with a
belief-based reward function in POMDPs. An important fact to note that this algorithm is able to
deal with any $\rho$-POMDP, and not only with Active Perception tasks. For example, this algorithm
can be applied to problems where the agent can influence the state of the world, and also (with
minor modifications to reintroduce state dependent rewards) when the reward function is both
dependent on states and belief.

\ys{you should explain here why POMCP cannot be applied to $\rho$ POMDP or any POMDP with belief dependent reward directly.}


\subsection{Non-PWLC Value Function}

Negative entropy is not a PWLC function. This means that the value function of a $\rho$-POMDP which
uses it as reward function will not be PWLC too \cite{cit:rpomdp}. This is a problem for offline
planning since usual solving techniques rely on the PWLC property in order to detect beliefs where
the currently computed optimal policy is not actually optimal. Once such a belief point is
discovered (called \textit{witness point}), the optimal policy is updated to be optimal in that
belief too.

This problem is solved in our approach since online planning does not suffer from that problem. This
is because online planning directly approximates rewards for the possible futures of any given
situation, without the need to solve a POMDP completely for any possible beliefs. Thus, witness
points do not need to be discovered, and we can ignore the PWLC property.


\ys{this is a potential source of confusion. $\rho$ POMDP has a PWLC value function and thus can be solved offline. This leads back to discussion we had about what is $\rho$POMDP and what is not. I know that the paper quotes that any POMPD with belief dependent reward is $\rho$ POMDP. However, the main distinctive property about $\rho$ POMDP is not that they were the first to define the reward in terms of $\rho$, but they were the first to take a $\rho$ defined reward and find a way to approximate it with PWLC function. This para sounds like it is not possible to solve $\rho$POMDP offline, and that is why we need your method, which is not true. You should use term 'POMDP with belief-dependent reward'}

\subsection{Reward Estimation}

The main challenge when applying POMCP to a $\rho$-POMDP is that the algorithm can only extract
reward information from a generative model. This cannot be done for a belief-dependent reward
function, since the belief is an abstraction created by the agent, which does not actually exist in
the environment.  Thus, a generative model is unable to sample from a belief-dependent reward
function.

\ys{Previous comments apply here too. $\rho$POMDP has a way of defining a belief dependent reward in terms of state.(which is what POMDP IR also does). However, in online planning, for any method that is not sampling based, it should be possible to compute future rewards directly from belief. In our case, since POMCP is a sampling based algorithm, we have to find a way of keeping estimates of belief-based rewards depending on the samples. Again this para sounds like POMCP cannot be applied to $\rho$POMDP, which is not true.}

To avoid this, we estimate the immediate reward for a specific action-observation transition by
using maximum-likelihood estimates of the belief extracted from the particle beliefs POMCP
generates. However, since particle beliefs are updated constantly, computing a direct estimate of
every belief reached during each sample episode would excessively hamper performance. We solved this
problem differently for estimating max of belief rewards and negative entropy rewards.

For estimation of max of belief, $\rho$-POMCP simply keeps track of the number of particles in each
particle belief, the number of particles for each type of particle and which type of particle is the
most common. At each particle insertion the number of particles for that type is compared against
the number of the old most common particle. Whichever is higher becomes the new most common
particle, and the new max of belief is computed as the ratio between the most common particle count
and the total number of particles in the belief.

\ys{is there a better way to differentiate particles and 'type of particle'. I can understand what you are trying to say, but the word particle causes too much confusion here.}

\begin{algorithm}[H]
    \caption{Max of Belief Reward Estimation}
    \SetKwFunction{update}{Update Estimate}

    \SetKwProg{main}{Algorithm}{}{}
    \main{\update}{
        \KwData{Particle Belief b, Particle s}
\nl     N(s) = N(s) + 1\;
\nl     \If{N(s) $>$ N(max\_s)}{
\nl         max\_s = s\;
        }
\nl     $\rho(b)$ = $\frac{N(max\_s)}{N(b)}$\;
    }
\end{algorithm}

For estimation of entropy, the situation is not quite so straightforward. Entropy estimation from
discrete samples is bound to be biased, and there is no way to remove this bias completely
\cite{cit:badentropy}. On the other hand, techniques to decrease bias are computationally heavy and
rely on usage of additional complex distributions to account for bias \cite{cit:entropyfixes}.
Since $\rho$-POMCP needed to be as fast as possible, we decided to try a more direct approach in the
hope that any inaccuracies would not hamper significantly the search process.

Negative entropy is computed as a summation of terms, one for each type of particle:

\[ -H(b) = \sum_s p(s) \log p(s) \]
\[ -H(b) \approx \sum_s \frac{n_s}{N} \log \frac{n_s}{N} \]

$\rho$-POMCP stores internally both the latest negative entropy estimate $-H(b)$, and each of the
separate terms that compose it. At each new particle insertion in a particle belief, we simply
remove from our previous full estimate the relevant term for the particle state, recompute the term
and add it again. This approach is not mathematically correct as, in theory, all terms would need to
be updated at each insertion. This is because each entropy term depends on the probability of such
term, which in turn depends on the total number of particles in the belief, which has just been
increased by one. In essence, we are overestimating negative entropy. Again, we would have bias
anyway no matter the strategy we used.

\begin{algorithm}[H]
    \caption{Negative Entropy Reward Estimation}
    \SetKwFunction{update}{Update Estimate}

    \SetKwProg{main}{Algorithm}{}{}
    \main{\update}{
        \KwData{Particle Belief b, Particle Type s}
\nl     $\rho(b)$ = $\rho(b)$ - $\rho(s)$\;
\nl     N(s) = N(s) + 1\;
\nl     p = $\frac{N(s)}{N(b)}$\;
\nl     $\rho(s)$ = p $\cdot$ log(p)\;
\nl     $\rho(b)$ = $\rho(b)$ + $\rho(s)$\;
    }

\end{algorithm}

In practice, this bias turns out not to be a problem as long as each type of particle is added with
relative frequency, so as to keep all terms that compose the overall negative entropy ``fresh''.
Such an approach is in any case guaranteed to converge to the true entropy in the limit, when the
number of samples tends to infinite. This maintains the POMCP property that leads to an optimal
solution in the limit.

\ys{explanation of computation of entropy estimates can be structure this way. 1.) We follow the following procudeure to compute entropy estimates. 2). However, this procedure causes bias and other problems. 3) Earlier results show that bias is difficult to get rid of and other problems. 4) In practice, bias does not matter as long as we have a lot of samples. Right now these points seems a bit mixed with each other. }

\subsection{Value Backup}

A second problem is dependent on how returns are approximated \ys{computed or approximated?} in POMCP. The value for a
belief-action pair $b,a$ \ys{isnt this just $Q(b,a)$} is computed as the sum of two different factors: the immediate reward
$\rho(b,a)$ plus a discounted term representing the expected return from then on.

\[ V(b,a) = \rho(b,a) + \gamma \cdot \sum_{o\in \Omega} O(o|b,a) V(b') \]

\ys{should be Q}

Recall that in a belief MDP the reward function is defined as:

\[ \rho(b,a) = \sum_{s\in S} R(s,a) b(s) \]

To approximate this value POMCP simply averages all results from $R(s,a)$ done using particles
extracted from the all beliefs and actions encountered during sample episodes.

\[ V(b,a) \approx \frac{\sum_{s \in S} n_s R(s,a)}{N} + \gamma \cdot \frac{\sum_{o\in \Omega} n_o f_o}{N} \]

Where $N$ is the number of times that action $a$ has been taken in belief $b$, $n_s$ the number of
times state $s$ has been sampled from $b$ when taking action $a$, $n_o$ is the number of times that
observation $o$ has been seen from $b$ and $a$ and $f_o$ is the average of all values backed up from
lower in the tree after observation $o$. In POMCP a backed up value is simply the discounted sum of
all sampled rewards down in the episode. As with MCTS, as long as the better actions are chosen more
often than the others, this procedure will converge to the true expected return.

\ys{this parts needs to be written a bit more formally and using a conventional notation. ex: V(b,a) is Q(b,a). It is very difficult to understand what $f_{o}$ is ??}

However, in our case, the reward function is structured in the following way:

\[ \rho(b,a) = \sum_{o\in \Omega} O(o | b, a) \rho(b') \]

This is because both max of belief and negative entropy rewards can be only extracted from a belief,
and do not depend on actions. Thus, the value for a belief-action pair becomes:

\[ V(b,a) = \sum_{o\in \Omega} O(o | b,a) \rho(b') + \gamma\sum_{o\in\Omega} O(o|b,a) V(b') \]

The problem is that we estimate $\rho(b')$ directly, rather than by averaging samples. At the same
time, $V(b')$ depends on reward estimates of future beliefs, which again are estimated directly.
This means that whenever a new estimate for any given $\rho(b)$ is computed, the new value needs to
substitute the previous one all the way up the chain, and cannot be simply averaged in as it
happened in POMCP.

We solved this problem by keeping the average mechanism present in POMCP, with the difference that
we average together fake datapoints built specifically to replace previous estimates. For
example, suppose we have a particular estimate $r$ for $\rho(b, a)$.

\[ \rho(b,a) \approx r \]

We can decompose this into all our estimates for all beliefs reachable from $b$ and $a$:

\[ \rho(b,a) \approx \frac{\sum_{o\in\Omega} n_o r_o}{N} \]

Suppose now we experience $b$ and $a$ again, and obtain observation $\tilde{o}$. We update our new
reward estimate for the new belief into $r'_{\tilde{o}}$. If we averaged normally, we would get:

\[ \rho(b,a) \approx \frac{ n_{\tilde{o}} r_{\tilde{o}} + r'_{\tilde{o}} +
\sum_{o \neq \tilde{o} \in \Omega} n_o r_o}{N+1} \]

This is not what we want, however. Instead, suppose we average in $n_{\tilde{o}}(r'_{\tilde{o}} -
r_{\tilde{o}}) + r'_{\tilde{o}}$:

\[ \rho(b,a) \approx \frac{ n_{\tilde{o}} r_{\tilde{o}} + n_{\tilde{o}}(r'_{\tilde{o}} -
r_{\tilde{o}}) + r'_{\tilde{o}} +
\sum_{o \neq \tilde{o} \in \Omega} n_o r_o}{N+1} \]

\[ \rho(b,a) \approx \frac{ n_{\tilde{o}} r_{\tilde{o}} + n_{\tilde{o}}r'_{\tilde{o}} -
        n_{\tilde{o}} r_{\tilde{o}} + r'_{\tilde{o}} +
\sum_{o \neq \tilde{o} \in \Omega} n_o r_o}{N+1} \]

\[ \rho(b,a) \approx \frac{ ( n_{\tilde{o}}+1) r'_{\tilde{o}} +
\sum_{o \neq \tilde{o} \in \Omega} n_o r_o}{N+1} \]

Which is what we wanted. Similarly, we can create fake reward points to backup in the tree so that
any $V(b)$ can be substituted with the more up-to-date $V'(b)$. The value backed up would be:

\[ N ( V'(b) - V(b) ) + V'(b) \]

\subsection{Max vs Mean}

The POMCP algorithm averages together all sampled rewards, and depends on UCT to let the tree
converge to the true optimum. This works since UTC samples the best actions infinitely more often in
the limit, which lets the final estimated value for it converge to the true value. However, this
procedure needs to sample many more times to make up for the fact that, at first, results for
suboptimal actions are averaged into estimates too.

POMCP inherits this behaviour from the original MCTS algorithm. However, in the original paper from
2006 a number of different approaches to merge together backed up values were tried
\cite{cit:mcts}. The authors decided to stick with using the mean operator to backup values since,
with the computational power available at the time, it gave better and more consistent results.
However, the authors also demonstrated that if enough samples are available, using a max operator
over actions actually improves performance.

We implemented both backup methods for $\rho$-POMCP in order to determine which one would perform
best under different conditions.

\subsection{Pseudocode}

Here we describe the full $\rho$-POMCP algorithm in more detail.

\begin{algorithm}[H]
    \caption{$\rho$-Partially Observable Monte-Carlo Planning}
\begin{multicols}{2}
    \SetKwFunction{pomcp}{$\rho$-POMCP}
    \SetKwFunction{simulate}{Simulate}

    \SetKwProg{main}{Algorithm}{}{}
    \main{\pomcp{b}}{
        \KwData{Belief b}
\nl     \While{enough time is available}{
\nl         s $\sim$ b\;
\nl         \simulate{\O, s, 0}\;
        }
\nl     \KwRet $\arg\max_a V(a)$\;
    }

    \setcounter{AlgoLine}{0}
    \SetKwProg{simul}{Procedure}{}{}
    \simul{\simulate{h, s, d}}{
        \KwData{History h, State s, Depth d}
\nl     N($h$) = N($h$) + 1\;
\nl     select action $a$ with UCT\;
\nl     ($s', o$) $\sim$ G($s,a$)\;
\nl     \eIf{T($h,a,o$) is \O}{
\nl         initialize T($h,a,o$)\;
\nl         newNode = true\;
        }{
\nl         newNode = false\;
        }
\nl     B($h,a,o$) = B($h,a,o$) $\bigcup \;\{ s' \}$\;
\nl     \eIf{$d < $ horizon and newNode == false}{
\nl         r = \simulate{$(h,a,o), s', d+1$}\;
        }{
\nl         N($h,a,o$) = N($h,a,o$) + 1\;
\nl         r = $\rho(h,a,o)$\;
        }
\nl     N($h,a$) = N($h,a$) + 1\;
\nl     V($h,a$) = V($h,a$) + $\frac{r - V(h,a)}{N(h,a)}$\;
\nl     oldV = V($h$)\;
\nl     V($h$) = $\rho(h)$ + $\gamma \cdot \max_{a'} V(h,a')$\;
\nl     \KwRet N($h$) * ( V($h$) - oldV ) + V($h$);
    }
\end{multicols}
\end{algorithm}

$\rho$-POMCP starts off as POMCP, by simulating episodes from the root of the tree.

Starting from the root, $\rho$-POMCP samples a single state $s$ from the root's particle belief.
Then, using UCT, it selects a state $a$, and uses it with $s$ to sample a new state $s'$, a reward
$r$ and an observation $o$. The root child representing the sequence $(a,o)$ is selected, or if it
did not exist, it is created. The algorithm then updates the particle belief of the child before
deciding whether to continue to descend the tree. While descending, POMCP appends $s'$ to the
particle belief of the selected child node. Then, it repeats the action selection and sampling
process using $s'$ as its $s$. This repeats until the algorithm is unable to descend further into
the tree, at which point the future reward is approximated as simply the reward of the last
selected child's belief.

The algorithm then updates the value of all actions taken during the descent, while backing up fake
datapoints to update all old reward estimates up in the tree.

Notice that $\rho$-POMCP lacks a rollout procedure. This is because $\rho$-POMCP has no way to
approximate rewards when outside the tree, and so skips random policy rollouts altogether.

\section{Multi Person}

One of the main limiting factors for Monte-Carlo online methods is the branching factor of the
tackled problem. While MCTS, POMCP and $\rho$-POMCP can deal with incredibly large state and
observation spaces, they cannot deal easily with a problem that leads to a vast number of branches
to be explored. This is because each new branch significantly increases the number of sample
episodes that need to be performed in order to obtain an accurate estimate for the best action.

The main branching factor in the problem we are considering, which is multi-sensor systems, is the
number of targets to be tracked. This is because each target moves independently, and thus from a
given state there are many more possible states reachable. Thus, the branching factor for a problem
is exponential in the number of targets.

In order to avoid dealing with this problem, we setup a system where we run concurrently many
$\rho$-POMCP instances. Each solver computes an individual policy for a single target only and
receives individual observations. This allows each solver to work on a relatively small tree,
without exploding the number of branches. In addition, this allows to work with a variable number of
targets, since it possible to dynamically adjust the number of online planners running at any time,
depending on the number of targets present in the environment.

We then estimate the final action to be performed by merging the separate results of all the
solvers.  From each solver we then obtain a value estimate for each available action; the final
action is then the one that maximizes the sum of all values across solvers. In order to obtain
consistent evaluations of all root actions so that we can compare them between solvers, we disable
UCT for the first depth level in the tree, and instead sample equally from each action.
