\section{Active Perception}

The Active perception task has been approached in numerous ways.

% \cite{cit:relworkneuralnet} approach the Active Perception task using reinforcement learning through
% a neural network. The neural network has the task of recognizing a particular visual pattern, which
% is observed through a visual sensor, which can only read portions of the pattern at any single time.
% The neural network has the ability of moving the sensor in various ways, obtaining visualizations of
% the pattern from different point of views. Their approach requires training for every recognized
% pattern
%
% PROBLEM: It is RL, we do planning

% \cite{cit:relworkconvex} tackle the problem of selecting $k$ sensors out of a pool of $n$ sensors.
% They approximate the information function defined from the usage of any $k$ sensors with a convex
% function, that can be then solved efficiently with linear solvers. Once this is done, the found
% solution is checked for improvements by testing possible swaps with the sensors that were not selected.
%
% PROBLEM: In the end I don't really tackle multi-sensor here..

% \cite{cit:relworklagrange} tackle multi-sensor system management, considering dynamic programming
% and linear solvers in order to balance the trade-offs between gathering information and energy costs
% due to communication.
%
% PROBLEM: I don't really get the paper, it's way too complex..

\ys{If possible, put the name of the authors and then use 'cite'. Ex: Kreucher et. al. [14] ...}
\ys{add related work on entropy estimation. A little bit about work from paninski} 
\ys{add a little bit more about earlier solutions to dynamic sensor selection problem 
\begin{itemize}
\item Approximate dynamic programming for communication-constrained sensor network management. Do they use closed-loop polcies? Check for that, point that as difference, also you use UCT for planning that is another difference. 
\item Sensor selection via convex optimization. The same points as above.
\end{itemize}
}
\cite{cit:relworktanks} tracks the movement of multiple targets using a scan sensor. To keep track
of each target position they use weighted Monte-Carlo particles which are updated using a model of
the environment. Action selection tries to maximize the expected Rényi information divergence to
reduce entropy with respect to the posterior distribution of targets. They extend their approach to
non-myopic solutions by direct enumeration of all possible outcomes. A key difference is that we
track each target separately, thus reducing the complexity of the overall posterior distribution of
targets. In addition, our usage of POMDPs to model the environment allows computations of non-myopic
strategies in a much more natural way than direct enumeration. 

\cite{cit:relworkentropy} examine a localization problem using direction-of-arrival sensors. They
analyze the usage of an entropy-based heuristic to select a single sensor at a time step, and show
that on average the heuristic is able to select the sensor which maximizes the mutual information
between target location and sensor observation. Their approach does not approximate heuristic
calculations, requiring $O(S^2)$ time to evaluate a single sensor, which becomes prohibitive when
the space to observe is very large. We avoid this cost by using Monte Carlo simulations to
approximate entropy estimations.

\cite{cit:relworkspaan} uses POMDPs to approach the multi-sensor system management problem, where
only a subset of $k$ sensors out of a pool of $n$ sensors can be selected. Their approach is based
on state-based reward functions, where the agent is rewarded for activating cameras with a target in
the field of view. Their approach focused on approximate offline planning for the model using
the SYMBOLIC PERSEUS algorithm, a variant of PERSEUS \cite{cit:perseus}. A crucial difference
is that our work does not rely on an offline planner, and thus does not have to recompute a solution
in case of a change, and can be more easily applied to much bigger models.

\section{Online POMDP Solvers}

One of the earliest explorations into online POMDP solvers has been been done by
\cite{cit:relworkonline1}, resulting in the Real-Time Belief Space Search (RTBSS) algorithm. This
algorithm creates a tree out of every possible result, and computes full belief updates and value
backups in order to determine the best actions at each step. In order to decrease computational
cost, a branch and bound strategy is employed to avoid exploring actions that have been proved not
useful. This strategy, however, can be used only by computing bounds offline, and then applying that
information to the leaves on the tree.

Other algorithms, such as BI-POMDP \cite{cit:relworkonlinebi} and AEMS \cite{cit:relworkonlineaems}
function similarly to the RTBSS algorithm, but rely on internal heuristics, rather than a branch and
bound strategy, to determine how to prune the tree to explore.

These algorithms differs importantly in that they try to solve the current belief perfectly. This
results in problems scaling when the number of possible actions and observations is high. By using
a Monte Carlo approach we avoid that problem by sampling, which allow our algorhtm to focus on
probable branches of the tree.

% An alternative to tree exploration approaches is the RTDP-BEL algorithm \cite{cit:relworkonlineq}.
% This algorithm learns progressively values for a discretized set of beliefs from direct interaction
% with the environment, using a Q-Learning type of approach.
%
% PROBLEM: RL

Our work in particular is based upon POMCP, an online solver that takes its strengths from
Monte-Carlo simulations \cite{cit:pomcp} which we discussed in Section \ref{ref:pomcp}. In
particular, POMCP follows the approach of Monte Carlo Tree Search, which progressively builds a tree
of outcomes by using a generative model to generate sample outcomes. This tree is then used to
approximate the values of all states. POMCP uses particle approximations of beliefs throughout this
tree, in order to avoid the expensive cost of computing belief updates at every action-observation
step. The difference with our approach is that POMCP cannot deal with belief dependent rewards, and
thus cannot evaluate policies which are tasked with increasing knowledge.

A good overview of existing POMDP online approaches can be found in \cite{cit:relworkonlineall}.
