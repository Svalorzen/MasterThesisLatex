\section{Partially Observable Markov Decision Process}\label{ref:pomdp}
\sectionmark{POMDP}

The \textit{Partially Observable Markov Decision Process} (POMDP) \cite{cit:pomdp} is framework
which allows to model the dynamics of an environment in order to produce a policy that fulfills a
given task. POMDPs structure reasoning around a probabilistic automaton \cite{cit:probautomata}. An
agent \textit{acts} upon the environment, which is in a particular \textit{state}. As a result, the
environment \textit{transitions} stochastically to another state, as defined by a transition
function. During the transition, the agent receives a certain \textit{reward} representing its
achievements toward the task at hand.

In a POMDP the agent cannot see directly the current state of the world. Instead, the agent receives
an \textit{observation} dependent on the underlying state during each transition, which may or may
not uniquely disambiguate it. The observations are used to keep a statistic over the states the
environment could be in at any given time.

The objective of the agent is to maximize the reward obtained by acting in the environment while
reasoning about the state uncertainties. In doing so it will accomplish the task that is encoded
within the POMDP model.

% \ys{transition function can be subsubsection ?? and so can be other bullets. }
%
% Eugenio: Shimon told me he didn't like too many subsections (before I had a subsection for each
% component of the POMDP - T, R, O, and so on..). So I'm not sure what to do. I've removed the old
% subsections, maybe I'll ask Shimon directly what subsections he would like so I can do it once and
% for all.

\ys{bullets are fine. No need to change.}
\ys{It is better to stick with one form of voice. I think the majority of earlier part was written in passive and next sentence is in active.}
We can define a POMDP as a tuple

\begin{equation}
 POMDP = <S,A,T,R,\Omega,O,h>
\end{equation}

The elements of the tuple are called \textit{state space}($S$), \textit{action space}($A$),
\textit{transition function}($T$), \textit{reward function}($R$), \textit{observation
space}($\Omega$), \textit{observation function}($O$) and \textit{horizon}($h$).

The state space contains all possible configurations the environment can take.  The environment
transitions between states discretely and instantaneously at each \textit{timestep}, which
represents the atomic unit of change within the environment.

\begin{equation}
 S = \{s_0, s_1, s_2, ..., s_{|S|}\}
\end{equation}

All states must have the \textit{Markov} property, which means that the evolution of the environment
must depend solely on the information encoded in its current state, and nothing else. The Markov
property is required in order to allow the agent to reason only about the current state, which makes
POMDPs tractable.

The action space contains all actions available to the agent. Each action simultaneously fulfills
two purposes: to influence the evolution of the environment, and to gather information regarding the
state of the environment.

\begin{equation}
 A = \{ a_0, a_1, a_2, ..., a_{|A|} \}
\end{equation}

The observation space contains all observations that the agent can receive from the environment; at
each timestep the agent receives a single observation from the environment which is dependent from
the action performed and the new state of the environment.

\begin{equation}
 \Omega = \{ o_0, o_1, o_2, ..., o_{|\Omega|} \}
\end{equation}

The transition function encodes the consequences of actions with respect to the environment states.
It specifies the probability that by performing an action $a$ in a state $s$, the
environment will transition to a new state $s'$.

\begin{equation}
 T(s, a, s') = Pr(s' | s, a)
\end{equation}

Where $Pr(s'|s,a)$ is the probability of the environment transitioning to $s'$ from $s$ when the
agent performs action $a$. By using this notation, we can show formally what the Markov property is
about. This property holds when

\begin{equation}
 Pr(s_{t+1} | s_{0}, a_{0}, s_{1}, a_{1}, ..., s_{t}, a_{t} ) = Pr(s_{t+1} | s_t, a_t ) \hspace{1cm} \forall s \in S
\end{equation}

where $t$ denotes the timestep where a particular state was reached.

The observation function represents the probability of obtaining a certain observation upon reaching
a given state $s'$ after taking action $a$.

\begin{equation}
 O(s', a, o) = Pr(o | s', a)
\end{equation}

Note that the state is the one of \textit{arrival}, not the one where the action is performed.

The agent goals are encoded within the reward function. This function is defined upon states and
actions, and can take values within the real domain:

\begin{equation}
 R(s, a, s') \in \mathbb{R}
\end{equation}

The reward function represents the level of appeal that a particular situation or outcome represents
in the modeled task. For example, one could allocate a certain reward for attempting to walk, and an
even greater reward for successful efforts. On the other hand, one could allocate a negative reward
for failed attempts. All these would influence the final behavior of the agent.

Finally, the horizon specifies for how many timesteps the agent is allowed to act on the
environment. The agent will then find the best strategy in order to maximize accrued reward within
this time constraint.

We can make a brief example of how the POMDP framework works within the context of multi-camera
tracking of a single target. The state space would be represented by each possible state of the
environment, thus each state would represent a possible location of the target. At each timestep,
the agent would take a single action, to guess the current position of the target and to collect a
noisy observation from one of the cameras available to it. At the same time, the target would move
to a different location, as specified by the transition function. In this problem the transition
function is independent of the action of the agent, since looking through a camera has no effect on
the movement of the target.

The observation would be collected with respect to the new position of the target and the camera
chosen by the agent, as specified by the observation function. The agent would also collect a reward
depending on its guess of the previous position of the target. The process would then repeat for
each remaining timestep.

\subsection{History and Belief}

The observation function does not allow the agent to disambiguate between states unambiguously.
Instead, what happens is called \textit{perceptual aliasing}: each obtained observation is
compatible with a subset of $S$.

The agent must reason about which of the possible underlying states of the environment are
compatible with all previously obtained observations in order to plan its moves.

Unfortunately, this implies that the agent is unable to reason about the present. While the
environment states maintain their Markov property, the agent cannot observe them. Instead it needs
to store internally its whole history of actions and observations in order to keep track of which
states are possible at any point during an episode. This long list of actions and observations is
called \textit{history}.

An alternative way of maintaining the history is keeping a \textit{belief} over the current state of
the environment. Such a belief is simply a probability vector in $|S|$ dimensions, where each
element denotes the probability that the environment is in a particular state. The belief is a
\textit{sufficient statistic} of the state of the environment.

At each new timestep, the agent's belief needs to be updated with the new knowledge gained by the
agent. The belief can be updated with the following formula:

\begin{equation}
 b'(s') = \frac{O(s', a, o)\sum_{s\in S}T(s,a,s')b(s)}{Pr(o|a,b)}
\end{equation}

The denominator can be treated as a normalizing factor that makes $b'$ correctly sum up to 1, since
it is a probability vector.

The belief allows for a reinterpretation of the POMDP framework. The belief can in fact be used in
lieu of a state, since it is possible to compute new transition and reward functions that depend on
beliefs rather than states. This transforms automatically the POMDP into a \textit{belief MDP}
\cite{cit:pomdp}, where the belief is now the state of the environment. An MDP
\cite{cit:suttonbarto} is a subset of POMDPs, where the agent always knows the state of the
environment.

The belief MDP is created by generating a new transition function $\tau$ and a \textit{belief
dependent reward} function $\rho$ as follows:

\begin{equation}
 \tau(b,a,b') = \sum_{o\in \Omega} Pr(b' | b, a, o) \sum_{s\in S} O(s,a,o) b(s)
\end{equation}

Where

\begin{equation}
Pr(b' | b, a, o) = \left\{
  \begin{array}{lr}
    1 \hspace{.6cm}\text{if the belief update with arguments $b,a,o$ returns $b'$}\\
    0 \hspace{.6cm}\text{otherwise}
  \end{array}
\right.
\end{equation}

Similarly:

\begin{equation}
 \rho(b,a) = \sum_{s\in S} R(s,a) b(s)
 \label{rhoeq}
\end{equation}

Although this transformation allows us to forget the partial observability, which is now embedded in
$\tau$, we have now introduced a new problem: a continuous state space. Although continuous MDPs are
very hard to solve generally, the belief MDP can be approached due to its specific properties.

\subsection{Policy}

In the POMDP framework the agent tries to maximize its \textit{expected return} within an
\textit{episode}: the expected sum of all rewards obtained within a single try at the task.

The length of a specific episode is determined by a parameter called \textit{horizon}. An horizon of
1 timestep generates what is generally known as a \textit{myopic policy}. Such a policy acts in a
locally optimal way, hoping that in doing so it will act in an optimal way globally. As the horizon
increases the final policy takes into account more and more possible outcomes, becoming more complex
but, depending on the task, more performant overall. We can describe the goal function of a POMDP
with finite horizon $h$ as:

\begin{equation}
 \mathbb{E}[\text{return}] = \mathbb{E} \left [ \sum_{t=0}^{h} R(s_t, a_t, s_{t+1}) \right ]
\end{equation}

A solution for a POMDP is called a \textit{policy}. We can define a policy $\pi$ as a function, that
given an horizon, a belief and action returns the probability of the action being chosen:

\begin{equation}
 \pi : H \times B \times A \rightarrow [0,1]
\end{equation}
\begin{equation}
 \pi(h, b, a) \in [0,1]
\end{equation}

We can also define it as a function, taking a belief and an horizon, returning a distribution over
actions:

\begin{equation}
 a \sim \pi(h, b)
\end{equation}

A policy performance can be evaluated through a \textit{value function}. The value function maps a
policy with its performance in terms of expected return from every possible belief the agent may
face and a particular horizon. Thus, every belief has a certain \textit{value} under a certain
policy and horizon. Formally then we can define the value of a belief as:

\begin{equation}
 V^\pi_{h}(b) = \mathbb{E}^\pi_h \left [\text{return} \mid b_0 = b \right ]
\end{equation}
\begin{equation}
 = \mathbb{E} \left [\sum_{t=0}^{h} \rho(b_{t}, a_t, b_{t+1}) \right ]
    \hspace{0.5cm}\text{where }b_0 = b \text{ and }a_t \sim \pi(h-t, b_t)
\end{equation}

This value is very important because it lets us compare and evaluate policies. This is because a
better policy does have a equal or higher value for every possible belief than an inferior policy
\cite{cit:suttonbarto}.

A value function can be expressed recursively with the \textit{Bellman equation}. This equation
states that the value of a state can be computed as the expected reward that the agent can obtain in
the next timestep, plus a value which represents what is going to happen in the future.

\begin{equation}
 V^{\pi}_{h}(b) = \sum_a \pi(h, b, a) \sum_{b'} \tau(b, a, b') \left [ \rho(b, a, b') +
V^{\pi}_{h-1}(b') \right ]
\end{equation}

Since the value function of a POMDP is a function dependent on beliefs, it is defined on a simplex
space of $|S|-1$ dimensions, which is continuous. The property that makes a belief MDP tractable is
that its value function is \textit{piecewise linear and convex} (PWLC). This means that it is
``determined by a set of hyperplanes, the value at a given belief point being that of the highest
hyperplane'' \cite{cit:rpomdp}. An example of this property is shown in Figure \ref{pwlcpic}. This
makes it possible to solve a belief MDP exactly, since the optimal policy is finitely bounded in the
number of components it is made of \cite{cit:pomdp}.

\begin{figure}[ht]
\begin{center}
\begin{tikzpicture}[>=stealth',auto,node distance=4cm,main node/.style={circle,draw,font=\Large\bfseries}]
\tikzstyle{state} = [circle, draw=black, fill=green!30]

\node[] at (0,-0.3) {$s_1$};
\node[] at (5,-0.3) {$s_2$};
\draw[step=0.5cm,lightgray,ultra thin] (0,0) grid (5,4.5);
\draw[thick, line cap=round] (0,0) -- (5,0) node[midway, anchor=north] {belief};
\draw[thick,->, line cap=round] (0,0) -- (0,5) node[midway, anchor=south, rotate=90] {Value};
\draw[thick,->, line cap=round] (5,0) -- (5,5) node[midway, anchor=north, rotate=90] {Value};
\draw[green] (0,3.5) -- (5,1) node[near start, anchor=south, sloped] {$a_1$};
\draw[blue] (0,1.5) -- (5,3.5) node[near start, anchor=south, sloped] {$a_2$};
\draw[red] (0,0.5) -- (5,4) node[near start, anchor=south, sloped] {$a_3$};

\end{tikzpicture}
\end{center}
\caption{An example of the PWLC property of the value function for a two state POMDP. Each line
represents an \textit{alpha vector}, which represents the value of a possible sequence of action in
any given possible belief. The value of the optimal policy is represented by the max of all possible
alpha vectors at any belief point, and thus is composed of a finite amount of linear components,
arranged in a convex shape.}
\label{pwlcpic}
\end{figure}

In general, when we are looking for a solution for a POMDP we are looking for a policy $\pi^*$, called
\textit{optimal policy}, that maximizes the value over all beliefs:

\begin{equation}
 V^{\pi^*}_h = \max_\pi V^{\pi}_h
\end{equation}

Another useful function that can be defined from $V$ is the value of a given action $a$ in a
particular belief, given that the specified policy is strictly followed after taking $a$:

\begin{equation}
 Q^{\pi}_h(b, a) = \sum_{b'} \tau(b,a,b') \left[ \rho(b,a,b') + V^{pi}_{h-1}(b') \right ]
\end{equation}

The \textit{Q-Function} is extremely useful since the optimal policy can be easily extracted from
it, simply by selecting at each belief the action that maximizes it. This work leverages this fact
by approximating the Q-function and then extracting the optimal policy from it.

\section{$\rho$POMDP}

As we have already mentioned, in a belief MDP the reward is defined in terms of the belief and the
normal reward function $R$, as shown in Equation \ref{rhoeq}.

However, in some cases the reward function is solely dependent on the information the agent has on
the current state of the world, rather than on the underlying state of the world itself. In such
situations the original POMDP reward function becomes useless, together with our definition of
$\rho$.  For example, in the case of surveillance, there is no inherent reward for the agent when
the environment is in a state rather than another. Instead, the agent actions should maximize the
knowledge that the agent has of the world.

In such cases the $\rho$ function is defined independently, and is not tied to states anymore. The
resulting framework is a variant of POMDP, called $\rho$POMDP, where $\rho$ substitutes $R$, while
all other elements of the original POMDP stay the same.

In the $\rho$POMDP framework unfortunately the PWLC property of the value function does not hold
anymore, and so the solving methods which rely on it cannot be used. While it has been proved in
\cite{cit:rpomdp} that as long as the newly defined $\rho$ function is PWLC then the value function
will be PWLC as well, functions such as entropy are not PWLC in the first place. We discuss how we
overcame this problem in Section \ref{ref:approach}.

% \ys{you should explain $\rho$POMDP here. What you have written here, does not even talk about what $\rho$POMDP is. It only outlines the problem which $\rho$POMDP solves.}
%
% Eugenio: I'm not sure I understand. As we have already said multiple times, \rhoPOMDP is the
% framework which replaces R with \rho. Thus I fell I have explained here adequately what it is
% (since it is relatively easy to explain after all).

