
\section{Markov Decision Process}\label{ref:mdp}

The world is complicated. In order to be able to take decisions, a computer system needs to
understand how it works, and how it responds to actions. In order to do this, we want to wrap all the
complexity of the world into a mathematical framework to allow us to reason in a structured way. The
\textit{Markov Decision Process} (MDP), named after Andrey Markov, was developed for this task.

This framework is a simpler version of POMDPs, which we introduce later in Section \ref{ref:pomdp}.
It is used to model problems where the whole environment is visible at any moment by the agent, and
without fault. In practice, it is not able to handle partial observability, but it is still useful
since it introduces elements which we use later anyway. Its advantage over competing approaches is
its ability to easily produce non-myopic strategies, which do not act greedily on the short term,
but aim to maximize a goal over the long term.

In this work we assume that there is no learning to be done; any problem we model is known in
advance fully by the decision making entity. This entity can thus \textit{plan} its strategy in
advance, rather than learn it. This strategy is called a \textit{policy}.

The way we are going to create the framework is to split our world into two separate components: the
\textit{agent} and the \textit{environment}. The former represents our computer system and its
decision making abilities, while the second represents everything else.

An MDP can be defined formally as a tuple

\[ MDP = <S,A,T,R,\gamma> \]

The elements of the tuple are called state space, action space, transition function, reward function
and discount factor. We introduce these terms in the following Sections.

\subsection{Agent}

The agent is at the core of the decision process. It has a goal: to affect the environment so that
some form of "goodness" within it is achieved. The agent does not need to be part of the
environment, or be physically present within it; it is an abstract decision making entity. The only
thing the agent needs is a way to act upon the environment, so as to steer the way it changes
over time.

%In essence, the agent represents everything that we are able to control in an \textit{absolute} way
%within the world we have to work with. Unfortunately, this is not much. To make a simple example, in
%the case of a robot most often the physical body of the robot is part of the environment, rather
%than the agent. The agent, or brain, of the robot then acts so that the physical body moves,
%hopefully as predicted. But since the body can be subject to failure and precise movements and
%sensors are required to check whether a motion completed successfully, we regard it as environment.

In the MDP framework this connection is achieved through \textit{actions}. Each action has a
specific effect on the environment, and the agent tries to choose at any moment the most correct
one.

\subsection{Environment}

The environment represents all that is not the agent. Unfortunately the real world is usually much
too complicated to be constrained into a mathematical framework, so some descriptive approximations
can be used instead. Generally, the environment is described at a level of abstraction that allows
to predict effects with reasonable accuracy for the task at hand, without the need to specify
needless details which would only impede the decision making process. Just as a person does not
consider interactions on a molecular level when making coffee, we want to do the same for our
description of our environment.

In this way, we want to be able to describe our environment in terms of its possible configurations.
Each configuration, denoted \textit{state}, represents a unique arrangement of the world at any one
time. For example, in chess, the environment can be completely described by the positions of all
pieces on the game board. The full set of states is denoted \textit{state space}, and is usually
referred to as $S$. This is the first element of the MDP tuple.

\[ S = \{s_0, s_1, s_2, ..., s_{|S|}\} \]

Note that we assume that all sets presented in this thesis work to be finite, unless where
specifically noted.

In order to describe how the environment is allowed to change over time, we can define
cause-effect links between states. This is effectively equivalent to a state machine.

%Just as in a state-machine, we can now start to reason in which ways the
%environment is allowed to change, and what we should do to pick the path that mostly suits our
%goals. These links need not be deterministic: they can, and will very often be, stochastic. This
%stochasticity can be used to represent a real stochasticity of the world underlying our abstract
%environment, or can be used to abstract away details that we know exist but are unwilling to model
%to reduce overall complexity. All in all, these two options are often much the same. In a dice roll,
%we could decide to model all physical forces acting on the dice, or just model the result as a
%stochastic number taken at random.

In the environment, transitions between a state and another state are discrete and instantaneous.
Each transition happens within a single \textit{timestep}, which represents the atomic unit of time
within the environment. Note that timesteps can be abstract, as they do not have to be directly
linked with time. Rather, a timestep is associated with environment changes that are significant to
the decision process, and so tie again to the concept of a state-machine. Whether those changes
happen in hours of real time or in seconds - or whether different changes occur with different
durations - does not influence the framework.

%A very important property of our environment, however, is that it is not allowed to contain any
%external decision-making entities. This is because such entities are themselves able to influence
%the evolution of the environment over time in ways our agent cannot predict, and modeling them
%though a state-machine framework is next to impossible. A whole branch of decision theory is
%dedicated to the study of such systems: multi-agent problems are by far more complex than
%single-agent ones.
%
%In our particular problem, prediction of people movement within a location, each person can be seen
%as an external entity which decides autonomously where it wants to move. Trying to model each and
%every person fully is a hopeless endeavor. At the same time, in the context of walking from a point
%A to a point B, people can be fairly predictable. Thankfully, evolution made us realize a long time
%ago that the fastest way to travel between two points is a straight line, and that can help in
%predicting where a person is going to go. Even though the path of each single person will be hard to
%predict with perfect accuracy, we can predict possible alternatives stochastically with a decent
%degree of accuracy, and that allows us to ignore the problem of multiple independent autonomous
%entities affecting the world simultaneously.

\subsection{State}

As we said above, our environment is described through states, which all together form the
\textit{state space} of the problem. Each state is unique, meaning that it represents a unique
circumstance of the environment, different in some particular way from all others. At the same time,
a state is a unique representation of the environment that ignores all that is not relevant to the
decision making progress. In a chess game, it does not matter whether at the time of playing it's
raining, but only the position of the pieces on the board.

%There are multiple ways of defining the states of the environment. One way is to enumerate them all.
%In this case, each state is in some sense independent from all others, and must be specified
%uniquely and added to the state space. This is called an \textit{extrinsic} definition of the state
%space. On the other hand, it is possible to define a number of features to define all states. In this
%way, a feature is a particular property of the environment that can take on a certain number of
%values. Following, a state is defined by the values all features are taking at any one time, and the
%whole state-space results from the product of all features. To keep with the chess analogy, one
%could describe every possible position by a combination of the position of every single piece - a
%feature. Such description of the state space is called \textit{intrinsic}.

An extremely important property that each state needs to have is called the \textit{Markov}
property. This property requires that the evolution of the environment depends only on the
information encoded in its current state, and nothing else. This can also be seen as saying that the
environment has no memory, since keeping track of previous states has no use for its future
evolution. The Markov property is required in order to allow the agent to reason only about the
current state, without the need to know its history.

While it can happen that a particular environment can evolve in a time-sensitive manner, it can be
easily shown that all such environments can be modeled in a Markovian way by simply adding more
features into the state-space definition \cite{cit:boutilier}.

\subsection{Action}

An action is a way in which the agent can influence the environment. Although in the problem that is
specifically considered in this work the agent has no control over the evolution of the environment
in the more general MDP framework actions can also be used to alter the way the environment can
evolve over time. In our problem we use a POMDP, which directly links actions to observations, where
the agent can actively collect information about the environment. Actions are thus relevant even
when the environment cannot be directly changed. We refer to Section \ref{ref:obsspace} for discussion
about this topic.

Actions can represent some actual action the agent is allowed to take, such as activating a motor,
or can express a more abstract function, such as a robot walking forward. These abstract actions can
then be performed by simpler subroutines which perform all the required intermediate steps to
complete the full action. The only requirement for the definition of an action is that its
consequences on the environment are understood.

The full set of actions available to the agent is called the \textit{action space}, or otherwise
referred to as $A$. This is the second element of the MDP tuple. It is usually assumed that all
actions are available to the agent at any timestep.

\[ A = \{ a_0, a_1, a_2, ..., a_{|A|} \} \]

\subsection{Transition Function}

In an MDP, the evolution dynamics of the environment are encoded in a function, called
\textit{transition function}. This function allows to specify the interrelations between each state
and each action, and the consequences of actions in terms of environment progression over time.

In particular the transition function is a function

\[ T: S\times A \times S \rightarrow [0,1] \]

that specifies the probability for any transition in the environment, depending on the actions that
the agent can take. In other words, it specifies the probability that by applying a particular
action in a particular state, we end up into a second state. Using this function, the agent can
determine the best way to reach a particular state from any other state. Another way to see this
function is to look at it as representing a conditional probability function:

\[ T(s, a, s') = Pr(s' | s, a) \]

Where $Pr(s'|s,a)$ is the probability of the environment transitioning to $s'$ from $s$ when the
agent performs action $a$. By using this notation, we can additionally show mathematically what the
Markov property is about. This property holds when

\[ Pr(s_{t+1} | s_{0}, a_{0}, s_{1}, a_{1}, ..., s_{t}, a_{t} ) = Pr(s_{t+1} | s_t, a_t ) \hspace{1cm} \forall s \in S \]

\subsection{Reward Function}

We use a similar function, called the \textit{reward} function, to encode the goals that we wish the
agent to achieve. This function is once again defined upon states and actions, but can take values
within the real domain:

\[ R: S\times A\times S \rightarrow \mathbb{R} \]
\[ R(s, a, s') \in \mathbb{R} \]

Encoding goals as numbers can initially be quite strange if someone is not used to it. What this
function represents is the level of appeal that a particular situation or outcome represents to the
agent. For example, we can specify whether a particular state of the environment is preferable over
another, and as such transitions that lead to the former are represented with a higher number in
the reward function. Or, we could prefer rewarding the agent for a particular effort, setting the
reward function to high values when a particular action is tried in a particular state. On the other
hand, negative terms can be used to inflict penalties over particular events that we do not want to
happen.

In these terms, all possible achievements and penalties have a weight and can be compared to each
other, in order to allow the agent to determine which goal to pursue depending on circumstances. The
usage of numbers is necessary in order for this weighting to take place at all.

Note that the reward function specifies a reward for each single transition the agent encounters,
independently from the current timestep. It occurs often that most of the transitions have no
particular reward, but the agent still needs to consider all of them when computing its policy.

After defining the reward function, one can start to imagine how the MDP framework operates. The
agent has access to actions, and it knows how these actions affect the environment via the
transition function. Using this function it can determine the best sequence of actions to select in
order to steer the environment towards a particular state. Finally, the state to reach is determined
by weighting pros and cons of any specific trajectory within the environment, and all intermediate
rewards the agent could pick up along the way.

\subsection{Discount and Horizon}

As we mentioned before, an MDP is a tool that takes into account future rewards in order to compute
the best possible policy. The agent does in fact try to maximize its \textit{expected return} within
an \textit{episode}: the expected sum of all rewards obtained within a single try at the task. We
can define it as a simple summation:

\[ \mathbb{E}[\text{return}] = \mathbb{E} \left [ \sum_{t=0}^h R(s_t, a_t, s_{t+1}) \right ] \]

An important fact is that the optimal policy is very dependent on the exact amount of time that the
agent thinks it's going to act within the environment. This time is called \textit{horizon}, and is
represented by $h$ in the above equation.

An horizon of 1 timestep generates what is generally known as a \textit{greedy policy}. Such a
policy acts in a locally optimal way, hoping that in doing so it will act in an optimal way
globally. As the horizon increases the final policy takes into account more and more possible
outcomes, becoming more complex but, depending on the task, more performant overall. On the opposite
side of a single step horizon, we have an \textit{infinite horizon}. Policies computed for infinite
horizons would be infinitely complex, but what happens in general is that policies tend to converge
as the horizon grows, so it is possible to have a very simple policy for an infinite horizon.

Infinite horizons have a disadvantage though. While in finite horizons the agent has an incentive as
to act optimally, in order to obtain the maximum possible reward within its alloted time, in an
infinite horizon there is no such urgency. Reward obtaining actions can be postponed indefinitely,
since in the limit the reward the agent will be able to obtain will remain the same (infinite).

A different way of presenting the concept of horizon is with the concept of \textit{discounting}.
Discounting makes rewards less valuable the further away in time they get collected. A simple
real-world example are banks interest rates: they lend money in the present, and to compensate for
that, they ask in return more money in the future. This is precisely because obtaining something now
is more valuable than obtaining it in the future.

An agent thus tries to maximize the \textit{expected discounted return}. All gained rewards are
decreased geometrically by a discount factor $\gamma \in [0,1]$:

\[ \mathbb{E}[\text{discounted return}] = \mathbb{E} \left [ \sum_{t=0}^h \gamma^t R(s_t,
a_t, s_{t+1}) \right ] \]

In this way, rewards that are far in the future become essentially meaningless, forcing the agent to
consider the present as more important of the far future. In essence, we are forcing the discounted
return to converge to a finite quantity. When $\gamma$ is 0 we have a single-step problem, where the
agent wants to maximize its currently available reward using a greedy policy.  When $\gamma$ is 1 we
have again the situation where there is no incentive for an agent to act, since it will still be
able to act in the future. $\gamma$ is the fifth and final element of the MDP tuple.

\subsection{Policy}

At this point we have introduced all elements that compose the MDP. Here we talk about the solution
itself, how it looks like and what it represents, while postponing the discussion about solving
algorithms for Section \ref{ref:solutions} onward.

A solution for an MDP is called a \textit{policy}. This policy specifies a sequence of actions to
be taken by the agent, depending on the states the agent will face while acting in the model. In
general we are interested in what is called the \textit{optimal policy}, which is the one whose
actions maximize the expected (discounted) return: it behaves better than all other policies on
average. Policies can be stochastic, so that in a given state there may be multiple action options
from where one is randomly picked up.

The policy's decisions take into account the chosen horizon for the problem. Thus, the policy's
decisions are dependent on the number of timesteps that the agent has still left to act. We can
define a policy $\pi$ as a function, that given an horizon, a state and action returns the
probability of the action being chosen:

\[ \pi : H \times S \times A \rightarrow [0,1] \]
\[ \pi(h, s, a) \in [0,1] \]

We can also define it as a function, taking a state and an horizon, returning a distribution over
action, which one can sample:

\[ a \sim \pi(h, s) \]

A useful property of the optimal policy is that is composed recursively. An optimal policy for
horizon $h$ only defines a single action to be taken per state. Once that action has been performed,
the agent can safely adopt the optimal policy for horizon $h-1$ from its resulting next state. This
means that any optimal policy for a particular horizon also contains implicitly all optimal policies
for all lower horizons.

\subsection{Value Function}

Since the environment, as encoded in the transition function, is stochastic, the agent cannot
predict with full accuracy which states it will experience during an episode, nor how much reward a
certain sequence of actions will be able to accumulate. What it can do instead is to compute the
expected return that will be obtained, as in the average cumulative reward that the agent would
collect if it was allowed to retry the same sequence of actions from the same state many, many
times. This expected return is of course computed with respect to a certain policy, since as we said
the reward obtained depends on which actions the agent performs.

The expected return depends on the state $s$ the environment is in where the agent needs to act.
Thus it is also called the \textit{value} of state $s$ with respect to, or under, some particular
policy $\pi$. This value of course depends also on how many steps the agent has left to take.
Formally then we can define the value of a state as:

\[ V^\pi_{s\:h} = \mathbb{E}^\pi_h \left [\text{discounted return} \mid s_0 = s \right ] \]
\[ = \mathbb{E} \left [\sum_{t=0}^{h} \gamma^t R(s_{t}, a_t, s_{t+1}) \right ]
    \hspace{0.5cm}\text{where }s_0 = s \text{ and }a_t \sim \pi(h-t, s_t) \]

This value is very important because it lets us compare and evaluate policies. A better policy does
in fact have a higher value or equal value for every possible state than an inferior policy.

An even more important property of value functions was discovered by Richard Bellman. It is a way of
representing a value function in terms of itself, recursively. The value of a state can thus be
defined as

\[ V^{\pi}_{s\:h} = \sum_a \pi(h, s, a) \sum_{s'} T(s, a, s') \left [ R(s, a, s') + \gamma
V^{\pi}_{s'\:h-1} \right ] \]

This is also known as the \textit{Bellman equation}. The value of a state can be thus computed as
the expected reward that the agent can obtain in the next timestep, plus a discounted total which
represents what is going to happen in the future. The fact that this equation is recursive is
immensely important and forms a base for many offline planning algorithms.

In general, when we are looking for a solution we are looking for a policy $\pi^*$that maximizes the value
of all states:

\[ V^{\pi^*}_h = \max_\pi V^{\pi}_h \]

\subsection{Q-function}

A function that is closely tied with the value function is the \textit{Q-function}. The Q-function
also contains values with respect to a policy and a state, but they also depends on an action. The
Q-function represents the value a state would have if we performed such an action, and followed the
specified policy from there after. In short:

\[ Q^{\pi}_{h}(s,a) = \sum_{s'} T(s, a, s') \left [ R(s, a, s') + \gamma
V^{\pi}_{s'\:h-1} \right ] \]

The Q-function also has an optimal version, $Q^*$, which depends, of course, on $V^*$. Extracting an
optimal policy from $Q^*$ is extremely easy, as it is possible to simply perform a maximization on
the actions over the Q-function, and the best action is automatically found.

\[ a^* = \arg\max_a Q^*_h(s,a) \hspace{.5cm}\forall s \in S\]

Many solution methods rely on the direct approximation of the optimal Q-function in order to find
the optimal policy. One of the beauties of this approach is the optimal policy can computed via a
single-step maximization of the Q-function. This is because the Q-function already encodes
internally the expectations over future rewards, and in this way allows us to perform a simple
greedy maximization to obtain a non-myopic solution.

\section{Partially Observable MDP}\label{ref:pomdp}

The framework we have described in Section \ref{ref:mdp} is used in order to model environments
where the agent has full access to state information during the full course of each an any episodes.
In an MDP, in fact, the agent is always able to reason simply in terms of the current state, and
take current decisions based on that information.

In the real world, however, most always an agent does not have access to the true underlying state
of the world. Instead what often happens is that the agent has access to imperfect sensors, which
rely incomplete information useful to understand in which state the environment is currently in. The
agent then needs to take into account the probability that the environment is in a state versus another
before deciding how to act.

Thus we extend the MDP framework into a \textit{Partially Observable Markov Decision Process}
(POMDP). In a POMDP all terms that we have defined previously are still used and taken into account,
but we are going to add on to these definitions a couple more.

A POMDP can be defined as a tuple $<S,A,T,R,\Omega,O,\gamma>$. We now define $\Omega$ and $O$.

\subsection{Observation}\label{ref:obsspace}

In a POMDP the agent does not have access to the current state of the environment. The state is
still there, and the environment is always in a particular state at any timestep, but the agent has
no direct way to know which one it is. The agent instead receives, at each timestep, an
\textit{observation}, which depends on the action that the agent performed and on the new state that
has been reached.

All observations are, as are states and actions, grouped in a set, $\Omega$, where usually $|\Omega| <
|S|$.

\[ \Omega = \{ o_0, o_1, o_2, ..., o_{|\Omega|} \} \]

\subsection{Observation Function}

The framework models the way the agent is able to receive information from observation as an
additional function added on top of the MDP framework. This function is called the
\textit{observation function}. It is a function

\[ O : S' \times A \times O \rightarrow [0,1] \]

so it relays the probability of receiving a certain observation upon reaching a given state $s'$
after taking action $a$.

\[ O(s', a, o) = Pr(o | s', a) \]

Note that the state is the one of \textit{arrival}, not the one where the action is performed.

\subsection{History and Belief}

The observation function does not allow the agent to disambiguate between states, or the problem
would convert directly to an MDP. Instead, what happens is called \textit{perceptual aliasing}: each
obtained observation is compatible with a subset of $S$. The agent must reason about which of the
possible underlying states of the environment are compatible with the observation obtained in order
to plan its moves.

Unfortunately, this implies that the agent is now unable to reason about the present. While the
environment states maintain their Markov property, the agent cannot observe them. Instead it needs
to store internally its whole history of actions and observations in order to keep track of which
states are possible at any point in an episode. This long list of actions and observations is called
\textit{history}.

An alternative way of maintaining the history is keeping a \textit{belief} over the current state of
the environment. Such a belief is simply a probability vector in $|S|$ dimensions, where each
element denotes the probability that the environment is in a particular state. Such a representation
becomes a very powerful abstraction that allows the computation of values with respect to beliefs.

At each new timestep, the agent's belief needs to be updated with the new knowledge gained by the
agent. The belief can be updated with the following formula:

\[ b'(s') = \frac{O(s', a, o)\sum_{s\in S}T(s,a,s')b(s)}{Pr(o|a,b)} \]

The denominator can be treated as a normalizing factor that makes $b'$ correctly sum up to 1, since
it is a probability vector.

\subsection{Belief MDP}

The belief allows for a reinterpretation of the POMDP framework. The belief can in fact be used in
lieu of a state, since it is possible to compute new transition and reward functions that depend on
beliefs rather than states. This transforms automatically the POMDP into an MDP, where the
belief is now the state of the agent.

We can show how this can be done using $\tau$ as the new transition function and $\rho$ as the new
reward function:

\[ \tau(b,a,b') = \sum_{o\in \Omega} Pr(b' | b, a, o) \sum_{s\in S} O(s,a,o) b(s) \]

Where

\[Pr(b' | b, a, o) = \left\{
  \begin{array}{lr}
    1 \hspace{.6cm}\text{if the belief update with arguments $b,a,o$ returns $b'$}\\
    0 \hspace{.6cm}\text{otherwise}
  \end{array}
\right.
\]

Similarly:

\[ \rho(b,a) = \sum_{s\in S} R(s,a) b(s) \]

Although this transformation allows us to forget the partial observability, we have now introduced a
new problem: a continuous state space. Although continuous MDPs are very hard to solve generally,
the belief MDP can be approached due to a property of its value function.

\subsection{Value Function}

In the MDP framework a value function needs to store a value for each possible state in the
environment. Since the number of states in the belief MDP is infinite, in theory a full value
function would be impossible to store, and the same would hold for any policy. However, in the case
of a belief MDP we can leverage a particular property of the value function that lets us avoid this
problem.

The value function of a POMDP is a function dependent on beliefs. Thus, the value function is
defined on a simplex space of $|S|-1$ dimensions. The property that makes POMDPs tractable is
that their value function is \textit{piecewise linear and convex} (PLC). In addition, the number of
polytopes of which it is composed has a finite bound at any horizon - exponential in the number of
possible observations.

These properties bound the number of optimal values that actually need to be stored, as only one
value for each linear polytope needs to be stored, making possible to solve a POMDP exactly. For a
more in-depth analysis of POMDP properties refer to \cite{cit:pomdp}.

\subsection{Policy}

A policy in a finite-horizon MDP is a set of sequences of actions that need to be taken, where each
such sequence is associated with a single state. This allows the agent to act no matter what it is
its starting state.

This is similar in a POMDP, with the difference that a different sequence of actions needs to be
stored for each linear polytope of the value function, instead that for each unique belief. The
optimal policy maintains the recursive property that we have defined previously for MDPs, where any
strategy for a particular horizon $h$ and belief $b$ is composed by a particular action plus the
optimal policy for horizon $h-1$.

%\section{Generative Model}% START FROM HERE
%
%What we have described up until now are models which assume that you have a lot of explicit
%knowledge about the problem you want to solve. This is because you need to know perfectly the
%transition function, observation function and reward function. Having this kind of intimate
%knowledge of a problem is, unfortunately, relatively rare.
%
%Thankfully, in order to apply Monte-Carlo techniques there is no need to have this level of detail
%in the description of a problem. Instead it is possible to work solely on a generative model which
%the agent can sample from in order to reproduce the behavior of the world, but is at the same time a
%black box with respect to its inner workings. Samples drawn from a generative model are
%indistinguishable from those sampled from a full-fledged model.
%
%Generative models are useful as one can gather real world data from an unknown stochastic process
%and use it to feed a Monte-Carlo planner, without the need to accurately study the process and
%determine exact probabilities and rewards for the transition and reward functions. It can also
%happen that computationally, a generative model is faster to run than a model which requires
%maintaining huge internal tables to store the transition and reward functions.
%
%In both cases, it is important to keep in mind that not all components of an MDP need to be directly
%available in order to be able to compute a good policy.

\section{$\rho$POMDP}

As we have already mentioned, in a POMDP the reward is defined in terms of the belief, as a linear
combination of the basic MDP reward function:

\[ \rho(b,a) = \sum_{s\in S} R(s,a) b(s) \]

However, in many cases, such as ours, this definition is limiting and cannot be used. If the reward
function is truly dependent on the information the agent has on the current state of the world,
rather than on the underlying state of the world itself, the MDP reward function becomes useless,
and as such our definition of $\rho$. In the case of surveillance, the agent actions should maximize
the knowledge that the agent has of the world, independently of which state the world is currently
in.

In this case, the $\rho$ function is defined independently, and not tied to states anymore.
Unfortunately, this means giving up on the PLC properties of the value function, which
were the only ways in which we were able to solve a POMDP in the first place. While it has
been proved in \cite{cit:rpomdp} that as long as the newly defined $\rho$ function is PLC then the
value function will be PLC as well, functions such as entropy are not PLC in the first place. We
discuss how we overcame this problem in Section \ref{ref:approach}.
