POMDPs can be used naturally to compute non-myopic policies to solve a particular task. We show here
that some Active Perception tasks require the usage of non-myopic solutions to achieve optimality,
and thus we can use POMDPs to model efficiently such tasks.

We propose two different POMDP models of Active Perception problems. For one of the models we will
prove that a non-myopic solution is required in order to obtain an optimal solution. For the second
we will prove that a non-myopic solution is not required. The results we show are for both max of
belief and negative entropy reward functions.

\section{Non-Myopic World}

Let's consider a world where the agent is tasked with tracking a single person that is walking. The
person can, at any moment, be in one of a series of rooms, and can transition between them. In each
room there is a camera which can observe the room itself and nothing else. If the agent activates
the camera situated in a room the person just entered there is a $0.8$ probability that the agent will
obtain a ``found'' observation, and a $0.2$ probability that it will obtain a ``null'' observation.
On the other hand, if the agent activates any other camera, it will obtain a ``null'' observation
with $0.8$ probability and a ``found'' observation with $0.2$ probability. Each time a camera is
activated, the person will move to another accessible room. The agent's goal is to guess correctly
where the person has moved for two times in a row, knowing only that at the start the person could
be in any room.

We will now create an instance of the problem we have just finished describing. Let's now suppose
that in our world there are 6 rooms, and we know that the person will transition from one to another
with the following probabilities:

\begin{center}
\begin{tikzpicture}[->,>=stealth',shorten >=1pt,auto,node distance=4cm,thick,main node/.style={circle,draw,font=\Large\bfseries}]
\tikzstyle{state} = [circle, draw=black, fill=green!30]
\tikzstyle{arrow} = [thick,->,>=stealth]

\node (s1) at (0,0) [state] {S1};
\node (s2) at (2,3) [state] {S2};
\node (s3) at (4,0) [state] {S3};
\node (s4) at (6,0) [state] {S4};
\node (s5) at (8,3) [state] {S5};
\node (s6) at (10,0) [state] {S6};

\path
    (s1) edge [loop below] node {0.333} (s1)
         edge [bend left] node {0.333} (s2)
         edge [bend left=10] node[below] {0.333} (s3)
    (s2) edge [loop above] node {0.333} (s2)
         edge [bend left=10] node[above,rotate=60] {0.333} (s1)
         edge [bend left] node {0.333} (s3)
    (s3) edge [loop below] node {0.333} (s3)
         edge [bend left=10] node[above,rotate=-60] {0.333} (s2)
         edge [bend left] node {0.333} (s1)
    (s4) edge [bend left] node {1.0} (s5)
    (s5) edge [bend left] node {1.0} (s6)
    (s6) edge [bend left] node {1.0} (s4);

\end{tikzpicture}
\end{center}

As we said, at first the agent does not know where the person is at all. The myopic expected return
for the first step only is:

\[ E[\text{return}| a_1 ] = \sum_{o\in \Omega} p(o | b, a_1) \cdot \rho(b') \]

Where $b'$ is the result of the belief update of $b$ with $o$ and $a_1$. The non-myopic expected return over
both steps is:

\[ E[\text{return}| a_1, a_2] = \sum_{o \in \Omega} p(o | b, a_1) \left ( \rho(b') + \sum_{o' \in
\Omega} p(o'| b', a_2) \cdot \rho(b'') \right ) \]

Where $b''$ updates from $b'$, $o'$ and $a_2$. It is easy to show that, both for max of belief and
negative entropy as $\rho$, all actions in the presented problem have the same myopic one-step
expected return from a uniform belief. This is done by simply computing the previously described
summatory over every term and performing the appropriate belief updates.

However, expected non-myopic returns are not the same for all choices of $a_1$ and $a_2$. For
example selecting $a_1 = S1$ always results in a certain expected return $v$, independently from
$a_2$. On the other hand, if $a_1 = S4$, all expected returns are greater than $v$ for all but one
choice of $a_2$. In other words, selecting $a_1 = S1$ leads necessarily to a lower two-step expected
return, since the agent can select $a_2$ to only pick the best cases.

Selecting $a_1$ to avoid the worst cases can only be done when there is a way to distinguish between
two different $a_1$.  Since myopic expected returns are the same for all $a_1$, this is impossible
to do with a myopic solution.

Note that the problem considered can be extended into uncountably many models where the same
properties apply. This could be done for example by adding a certain number of rooms that the person
needs to go through before ending up in the model presented, and so on. Thus a non-myopic solution
is required for uncountably many Active Perception tasks.

\section{Myopic World}

A different instance of the previous problem is one where there are only 4 rooms, and the person
will transition from state to state with the following probabilities:

\begin{center}
\begin{tikzpicture}[->,>=stealth',shorten >=1pt,auto,node distance=4cm,thick,main node/.style={circle,draw,font=\Large\bfseries}]
\tikzstyle{state} = [circle, draw=black, fill=green!30]
\tikzstyle{arrow} = [thick,->,>=stealth]

\node (s1) at (0,0) [state] {S1};
\node (s2) at (0,4) [state] {S2};
\node (s3) at (4,4) [state] {S3};
\node (s4) at (4,0) [state] {S4};

\path
    (s1) edge [bend right] node {0.5} (s4)
         edge [bend left] node {0.5} (s2)
    (s2) edge [bend left] node {0.5} (s1)
         edge [bend right] node {0.5} (s3)
    (s3) edge [bend right] node {0.5} (s2)
         edge [bend left] node {0.5} (s4)
    (s4) edge [bend left] node {0.5} (s3)
         edge [bend right] node {0.5} (s1);

\end{tikzpicture}
\end{center}

Once again, we can compute both myopic and non-myopic expected returns. This time, no matter the
choice of actions, there is no particular incentive for an agent to select a pair over another. If
$\rho$ is negative entropy, certain pair of actions have a higher expected return than others. But
for each $a_1$ there is a $a_2$ which obtains the maximum expected return possible.

Thus, non-myopic in this particular model has no particular advantage over a more simple myopic
solution. Once again this problem can be extended into uncountably infinite examples of Active
Perception tasks.

